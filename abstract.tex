\chapter*{Abstract}
\chaptermark{Abstract}
%\pagenumbering{roman}
%\newpage
%\begin{center}  
%	{\Huge \textbf{ Abstract}}
%\end{center}
\addcontentsline{toc}{chapter}{\textit{Abstract}} 
The Human Genome Project, launched to determine the molecular sequence of human DNA, was completed in 2003. Since then the period has been referred as the post-genomic era to date. In this period the amount of observed biological sequence data are being increasing to a great extent. Robust computational methods, frameworks, tools, etc., able to tackle practical challenges, need to be developed in the same pace to maximize the useful information extracted from those data. One of the fundamental concepts derived from biological sequences is the evolutionary relationships among a group of organisms particularly known as the phylogenetic tree. Such trees can offer crucial biological applications, for instance tracking the evolution of a disease, designing new drugs, etc. Multiple sequence alignment (MSA) is an important step in the pipeline of inferring the phylogenetic tree where the given sequences are arranged according to evolutionary history. The characteristics as well as the quality of the MSA obtained dramatically influences the accuracy of the estimated tree. The computation of MSA is an NP-hard optimization 
%in 2003 

%Nowadays, the increasing private vehicles have caused severe traffic congestion, environmental pollution and road accidents in many cities around the world. As such, public transport has been widely recognized as an effective way to improve urban life. To develop public bus service as a competitive alternative to private vehicles, we must design a practical, efficient and economic transit network. A transit network is composed of several connected routes for public buses. Transit Network Design Problem (TNDP) determines the transit network for a city while achieving some objectives and maintaining some constraints. In this modern age, TNDP involves different stakeholders with different interests and values. As a result, a large number of optimization objectives arise naturally. However, most of the researchers have ignored this issue by using single objective function to express solution quality. 
%In this thesis, we proposed a new formulation for the many-objective TNDP which allows to generate a diverse set of alternative solutions. Then we developed problem specific genetic operators for solving TNDP using evolutionary algorithms. For the first time, we adapted several state-of-the-art many-objective evolutionary algorithms (MaOEAs) to explore the high-dimensional objective space of TNDP using our genetic operators. Our MaOEA based approach has been rigorously tested with several benchmark datasets. The experimental results show that, the proposed methodology is more effective in addressing modern challenges than the existing approaches.






