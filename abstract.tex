\chapter*{Abstract}
\chaptermark{Abstract}
%\pagenumbering{roman}
%\newpage
%\begin{center}  
%	{\Huge \textbf{ Abstract}}
%\end{center}
\addcontentsline{toc}{chapter}{\textit{Abstract}} 
The Human Genome Project, launched to determine the molecular sequence of human DNA, was completed in 2003. Since then the period has been referred as the post-genomic era to date. In this period the amount of observed genomic data are being increasing to a great extent due to the availability and relative affordability of various sequencing technologies. Robust computational methods, frameworks, tools, etc., able to tackle practical challenges, need to be developed in the same pace to maximize the useful information extracted from those data. One of the fundamental concepts derived from biological sequences is the evolutionary relationships among a group of organisms particularly known as the phylogenetic tree. Such trees can offer crucial biological applications, for instance tracking the evolution of a disease, designing new drugs, etc. Multiple sequence alignment (MSA) is an important step in the pipeline of inferring the phylogenetic tree where the given sequences are arranged according to evolutionary history. The characteristics as well as the quality of the MSA obtained dramatically influences the accuracy of the estimated tree. 

Usually the MSAs are inferred by optimizing a single function or objective. The alignments estimated under one criterion may be different from the alignments generated by other criteria, inferring discordant homologies and thus leading to different hypothesized evolutionary histories relating the sequences. In recent past, researchers have advocated for the multi-objective (MO) optimization to address this issue, where multiple conflicting objective functions are being optimized simultaneously to generate a set of alternative alignments. However, no theoretical or empirical justification with respect to a real-life application has been shown for a particular MO formulation. In this thesis, we investigate the impact of MO formulation in the context of phylogenetic tree estimation. Employing MO metaheuristics, we demonstrate that (a) trees estimated on the alignments generated by MO formulation are substantially better than the trees estimated on the alignments generated by the state-of-the-art MSA tools and (b) highly accurate alignments with respect to popular measures do not necessarily lead to highly accurate phylogenetic trees. Thus, in essence, we ask the question whether an application-aware (in this case phylogeny-aware) metric can guide us in choosing appropriate MO formulations that can result in better phylogeny estimation. %And, we answer the question affirmatively through carefully designed extensive empirical study.

PASTA (Practical Alignments using SAT\'e and TrAnsitivity) is a state-of-the-art method for computing MSAs, well-known for its accuracy and scalability. It iteratively co-estimates both MSA and maximum likelihood (ML) phylogenetic tree. It attempts to exploit the close association between the accuracy of an MSA and the corresponding tree in finding the output through multiple iterations from both directions. Currently, PASTA uses the ML score as its sole optimization criterion which is a good score in phylogeny estimation but cannot be proven as a necessary and sufficient criterion to produce an accurate phylogenetic tree. Therefore the integration of multiple application-aware objectives, carefully chosen considering better association to the tree accuracy, into PASTA may potentially have a profound positive impact on its performance. In the sequel of this thesis, we employed the four application-aware objectives, identified earlier, alongside ML score to develop an MO framework, namely, PMAO, that leverages PASTA to generate a bunch of high-quality solutions that are considered equivalent in the context of conflicting objectives under consideration. We analyze this tree-space based on the tree generated by PASTA by experimenting on a popular biological benchmark and show that the tree-space contains significantly better trees than PASTA. To help the domain expert further in choosing the most appropriate tree from the PMAO output (containing a relatively large set of high-quality solutions), we incorporated a machine learning approach within the PMAO framework that is capable of generating a smaller set of high-quality solutions. %Additionally, we attempted to obtain a single high-quality solution without using any external evidence and found that summarizing the few solutions detected through machine learning can serve this purpose to some extent. 

MUSCLE is a general-purpose MSA tool widely used for its high throughput and accuracy. Carefully equipping MUSCLE with multiple application-aware objectives positively impacts its capability to yield better trees. This thesis introduce MAMMLE, a framework for inferring better phylogenetic trees from unaligned sequences by hybridizing MUSCLE with multiobjective optimization strategy and leveraging multiple Maximum Likelihood hypotheses. MAMMLE is an end-to-end approach for phylogeny estimation from unaligned sequences as a flexible framework whose components can potentially be modified, replaced, or further refined by bioinformatics researchers and practitioners. We provide the Linux and Mac OS X implementation of MAMMLE as an Open Source tool. We show that MAMMLE, in its basic form, can offer a significant improvement in tree accuracy over MUSCLE.

Species tree estimation from multi-locus data is complicated as biological processes can result in different loci having different evolutionary histories. Incomplete lineage sorting (ILS), modeled by the multi-species coalescent (MSC), is considered to be a dominant cause for gene tree incongruence. Various optimization criteria (e.g., quartet score, pseudo-likelihood, etc.) are statistically consistent under the MSC model, meaning that they return the true species tree with high probability given sufficiently large numbers of accurate gene trees. However, the number of genes is limited and estimating highly accurate gene trees is difficult. Therefore, even the statistically consistent methods may fail to reconstruct highly accurate trees under practical model conditions with limited numbers of genes and in the presence of gene tree estimation error. In this thesis, we present a MO metaheuristics algorithm (SNOGA), a modified version of the popular NSGAII, which combines various optimization criteria to find a suitable search space containing highly accurate species trees. Our experimental results on a collection of simulated datasets demonstrate that the MO approach can lead us to a tree-space containing significantly better trees than the trees estimated by ASTRAL and MP-EST which are two of the most widely used methods. 

%The computation of MSA is an NP-hard optimization 
%in 2003 








