%\addtocounter{chapter}{3}%only temporary
\chapter{Experimental Studies}
\label{chapter4}

\section{Introduction}
To evaluate any MaOEA it is necessary to simulate it on benchmark datasets and compare the obtained results with other approaches. In this chapter we evaluate our proposed methodology. At first we describe the datasets and performance metrics used for this purpose. We propose two new metrics for comparing our approach with previous ones. Then we analyze the effect of varying different components of our evolutionary framework. Finally, we compare our MaOEAs with several well known studies based on generated solutions.


\section{The Dataset}
A dataset provides the key inputs of TNDP which are a road network and its associated demand matrix. We run an algorithm to generate transit networks from those data. We can infer the performance of an algorithm by analyzing the properties of generated transit networks. In this study, we selected three datasets to examine different aspects of our proposed methodology. Visual representations of these datasets are presented in Figure~\ref{fig:datasets}. Table~\ref{tab:sum_datasets} gives a summary of these datasets. 

\begin{figure}[!htbp]
	\centering
	\begin{subfigure}[b]{0.33\textwidth}
		\includegraphics[width=\textwidth]{Figure/mandl2}
		\caption{Mandl}
		%\label{fig:con_pr06}
	\end{subfigure}%
	\begin{subfigure}[b]{0.33\textwidth}
		\includegraphics[width=\textwidth]{Figure/m0}
		\caption{Mumford0}
		%\label{fig:con_pr07}
	\end{subfigure}%
	\begin{subfigure}[b]{0.33\textwidth}
		\includegraphics[width=\textwidth]{Figure/m1}
		\caption{Mumford1}
		%\label{fig:con_pr09}
	\end{subfigure}%
	\caption{Visual representation of used datasets.}
	\label{fig:datasets}
\end{figure}

\begin{table}[!htbp]
	\centering
	\caption{Summary of datasets used in this study}
	\begin{tabular}{|c|r|r|r|c|}
		\hline
		Dataset & Nodes \& Edges & No. of Routes & Nodes/Routes & Trips\\
		\hline
		Mandl's swiss network & \multicolumn{1}{c|}{15 \& 20} & \multicolumn{1}{c|}{4, 6} & \multicolumn{1}{c|}{2 - 8} & 15,570 \\
		\hline
		Mumford0 & \multicolumn{1}{c|}{30 \& 90 } & \multicolumn{1}{c|}{12} & \multicolumn{1}{c|}{2 - 15} & 3,42,160 \\
		\hline
		Mumford1 & \multicolumn{1}{c|}{70 \& 210} & \multicolumn{1}{c|}{15} & \multicolumn{1}{c|}{10 - 30} & 19,26,170 \\
		\hline
	\end{tabular}%
	\label{tab:sum_datasets}%
\end{table}%

Mandl's swiss network~\cite{mandl1980evaluation} is currently the only available benchmark. It represents a real case study network of Switzerland. It consists of 15 nodes and 21 edges with a total demand of 15,570 trips per hour. Almost all researchers utilized this dataset to demonstrate the effectiveness of their approaches. This enabled us to compare our results with previous approaches. But the limitation is that, it is too small in size to represent real-world road networks. So it can provide a poor estimate for the effectiveness of an algorithm for larger datasets. A good algorithm should equally perform for large networks. However many researchers neglected this fact. 

To demonstrate the response to increase in dataset size, we tested our algorithms on two large networks, Mumford0 and Mumford1, proposed by Mumford~\cite{mumford2013new}. Mumford0 has 30 nodes connected by 90 edges and the total demand is 3,42,160 trips. Mumford1 consists of 70 nodes and 210 edges with a total demand of 19,26,170 trips. While Mumford0 is a synthetic dataset, Mumford1 is based on information manually extracted from bus route network maps obtained for Yubei, a city of China. Several other large datasets available in the literature. We chose these two datasets over others because they were utilized by several researchers. So we can compare our results with them.

\section{Performance Metric}\label{sec:pMes}

%GIVE CITATIONS HERE

Performance assessment is an important issue in evolutionary many-objective optimization. A number of performance metrics have been used to assess the performance of MaOEAs for a MaOP such as Generational Distance, Inverse Generational Distance, Spread and Hypervolume. The performance of a MaOEA is evaluated from two aspects: convergence and diversity. Inverse Generational Distance (IGD) and Hypervolume (HV) evaluate both of them in a combined sense. However IGD calculation requires a good knowledge of the true PF which is unrealistic. That's why, in this study we compare the capability of our MaOEAs in terms of HV. Later we propose two new metrics for comparing MaOEAs with previous approaches. 

\subsection{Hypervolume}

As mentioned in Section \ref{sec:MaOEA_background} that, the performance of any evolutionary algorithm for a MaOP is measured from two aspects: convergence and diversity. While convergence indicates the closeness of the obtained solution set to the PF, diversity determine the spread of solutions in $A$ across the $m$-dimensional objective space. Hypervolume (HV)~\cite{zitzler1999multiobjective} is a widely used performance metric that capture in one scalar both convergence and diversity. HV is a fair indicator for comparing different MaOEAs.

\begin{figure} [!htbp]
	\centering
	\includegraphics[width=6cm]{Figure/hv}\\
	\caption{HV enclosed by the non-dominated solutions~\cite{deb2001multi}.}\label{fig:hv} 
\end{figure}

HV of a set of solutions measures the volume of the portion of an objective space which is dominated by those solutions collectively. It has good mathematical properties and is the only metric known to be strictly Pareto-compliant~\cite{zitzler2003performance}.  Moreover, calculation of HV does not depend much on the PF. Thus HV is specially suitable for dealing with real-world MaOPs where the true PF is unknown. For a non-dominated solution set, HV is calculated with respect to a reference point. Let $A$ is the set of the non-dominated solutions obtained by an algorithm and {\bf r} = $(r_1,r_2,\cdots,r_m)^T$ is a $m$-dimensional reference point in the objective space, which is dominated by any solution in  $A$. Then HV of $A$ with respect to {\bf r} is the volume of region dominated by $A$ and bounded by {\bf r}. It can be defined as
\begin{equation}
HV(A,{\bf r}) = volume(\bigcup\limits_{f\in A}[f_1,r_1]\times\cdots\times[f_m,r_m])
\label{eq:hv}
\end{equation}
As the objective values of TNDP are differently scaled, we first normalize the objective values of the points in $A$ using ${\bf z}^{nad}$ (nadir point) and ${\bf z}^{*}$ (ideal point) before calculating HV. Then we set {\bf r} to $(1, 1, 1, 1, 1, 1, 1)^T$. The points which do not dominate {\bf r} are discarded to compute HV. An example of HV calculation is shown in Figure~\ref{fig:hv}. Here HV is represented by the hatched region. For minimization problem like ours, a large value of HV is desirable. 


\subsection{Proposed Metric}
\label{sec:new_metric}

We find that most of the previous works on TNDP used a single objective function to measure solution quality. Whereas in this study we use seven objective functions simultaneity. So it is not straightforward to compare the effectiveness of our approach with previous studies. Therefore we propose two new metrics to compare MaOEA based approaches with other approaches based on best solutions reported by the researchers for the benchmark datasets.  To calculate these metrics, we evaluated the objective vectors for those best solutions using our proposed evaluation model.

Let $A$ be the set of non-dominated solutions in the objective space generated by a MaOEA and $s$ be the objective vector representing the best solution generated by a previous approach for the same problem instance. With these two notations, we now define two new metrics as follows:

\begin{itemize}
	\item \textbf{Generated Dominating Solutions (GDS)} of a MaOEA with respect to $s$ is the number of solutions in $A$ that can dominate $s$. Mathematically 
	\begin{equation}
	GDS(A,s) = |\{a \in A \, | \, a \text{ dominates } s\}|
	\label{eq:gds}
	\end{equation}

	\item \textbf{Generated Alternative Solutions (GAS)} of a MaOEA with respect to $s$ is the number of solutions in $A$ that can be used as alternatives to $s$. We consider $a$ as an alternative to $s$ if the sum of results, obtained by subtracting normalized objective values of $s$ from $a$, is negative. Mathematically 
	\begin{equation}
	GAS(A,s) = |\{a \in A \, | \, \sum_{m=1}^{M}\frac{a_m - s_m - z_m^{*}}{z_m^{nad} - z_m^{*}} < 0  \}|
	\label{eq:gas}
	\end{equation}
\end{itemize}

A large value of both these metrics is desirable. As in many-objective space it is difficult to dominate a solution, the value of GDS can be zero for a particular previous solution. In that case the value of GAS can be helpful for comparison.     
%\subsection{Statistical Analysis}
%
%We used Wilcoxon signed-rank test~
%\cite{wilcoxon1945individual} for statistical hypothesis test on the experimental results. It is a popular non-parametric statistical hypothesis test which is an alternative to the paired Student's t-test. In this test, each pair of samples are first compared with each other. Based on the difference of these samples, a ranking is provided. Tied pairs are ignored. Based on these ranks and the difference, a sum of of the signed ranks is made. Then, using empirical data we can find if the null hypothesis, which states that there is no relationship between two measured phenomena, can be rejected. The whole process is discussed as follows.
%
%Let $N$ be the number of pairs. Also for $i=1,~ \ldots,~ N$, let $x_{1,i}$ and $x_{2,i}$ denote the measurements. Now we calculate $|x_{2,i}-x_{1,i}|$ and sgn($x_{2,i}-x_{1,i}$) for all pairs where sgn is the sign function. Pairs with $|x_{2,i}-x_{1,i}|=0$ are excluded. Let the reduced number of pairs be $N_r$. These pairs are then ordered according to their absolute difference value from smallest to largest. They are then ranked. Let the rank of pair $i$ be denoted by $R_i$. Now, $W$ is calculated as $W=|\sum_{i=1}^{N_r}[\text{sgn}(x_{2,i}-x_{1,i})\times R_i]|$. The sampling distribution of $W$ converges to a normal distribution with increasing $N_r$. When $N_r \geq 10$, a $z$-score can be calculated as $z=\frac{W-0.5}{\sigma_w}$ where $\sigma_w=\sqrt{\frac{N_r(N_r+1)(2N_r+1)}{6}}$. The null hypothesis can be rejected if this $z$-score is greater than a certain critical value. If, $N_r<10$, $W$ is compared with a critical value from a reference table.

\section{Experimental Setup} \label{sec:eSetup}
We conducted extensive experiments with four MaOEAs. The parameters of our MaOEAs with corresponding values are listed in Table~\ref{tab:parameters}. For each MaOEA we created 30 $(6 \times 5)$ variants by changing the crossover rate (six values) and mutation scheme (five options). Each variant was run independently for 20 times on each problem instance except Mumford1. For Mumford1, the largest network, we have performed 15 independent runs to cope with the huge increase in running time. We set the terminating criterion to 500 generations for each run. We implemented all the MaOEAs in JAVA using JMetal framework. The source codes along with datasets and statistical results are publicly available at \url{https://github.com/ali-nayeem/JMetal4.5_Netbeans}.


% Table generated by Excel2LaTeX from sheet 'param'
\begin{table}[!htbp]
	\centering
	\caption{Parameters of our algorithms}
	\begin{tabular}{|c|r|c|}
		\hline
		Algorithm & \multicolumn{1}{c|}{Parameter} & Value \\
		\hline
		\multirow{5}[10]{*}{All} & \multicolumn{1}{l|}{Number of runs} & 20 \\
		\cline{2-3}          & \multicolumn{1}{l|}{Population size, $N$} & 112 \\
		\cline{2-3}          & \multicolumn{1}{l|}{Maximum generations, $T$} & 500 \\
		\cline{2-3}          & \multicolumn{1}{l|}{Mutation scheme} & Algorithm~\ref{alg:basic_scheme}, \ref{alg:random_schemei}, \ref{alg:random_schemeii}, \ref{alg:random_schemeiii}, \ref{alg:guided_scheme} \\
		\cline{2-3}          & \multicolumn{1}{l|}{Crossover rate, $C_r$} & 0.0, 0.2, 0.4, 0.6, 0.8, 1.0 \\
		\hline
		SPEA2 & \multicolumn{1}{l|}{Archive size, $\overline{N}$} & 112 \\
		\hline
		\multirow{2}[4]{*}{MOEA/D} & \multicolumn{1}{l|}{Neighborhood size, $T'$} & 20 \\
		\cline{2-3}          & \multicolumn{1}{l|}{Probability of mating with neighbors, $\delta$} & 0.9 \\
		\hline
		NSGAIII & \multicolumn{1}{l|}{Number of reference points} & 112 \\
		\hline
		$\theta$-DEA & \multicolumn{1}{l|}{Number of reference points} & 112 \\
		\hline
	\end{tabular}%
	\label{tab:parameters}%
\end{table}%

\section{Result}

In this section we present our experimental results. We begin by visualizing the approximated Pareto Front (PF) of each problem instance constructed by extensive experimentation. Next we discuss the effect of our crossover operator on algorithm performance measured in terms of HV. Afterwards we analyze the HV results of our mutation schemes. We conclude by comparing our approach with previous ones using GDS and GAS. 
\subsection{Approximated Pareto Front}
For any MaOP, the PF can play a vital role in designing a robust optimization method. It reveals many important properties of the chosen objective functions. Analyzing this information we can reduce the complexity of a MaOP. Moreover PF provides a reference frame for measuring the optimization progress as well as the performance of a MaOEA. For example, to calculate HV we need to normalize the obtained solutions using $z^*$ and $z^{nad}$. These two points are obtained from the PF. However, the true PF can only be available for test problems. It is unknown for real-world MaOPs like TNDP.  
% Table generated by Excel2LaTeX from sheet 'abbr'
\begin{table}[!htbp]
	\centering
	\caption{Abbreviation for objective functions}
	\begin{tabular}{|l|c|}
		\hline
		Objective function & Abbr. \\
		\hline
		In-vehicle travel time & IVTT \\
		\hline
		Waiting time & WT \\
		\hline
		Percentage of transfer & TP \\
		\hline
		Percentage of unsatisfied demand & UP \\
		\hline
		Fleet size & FS \\
		\hline
		Route length & RL \\
		\hline
		Degree of route overlap & DO \\
		\hline
	\end{tabular}%
	\label{tab:abbr}%
\end{table}%

In this study we constructed an approximated PF for each problem instance for the the vindication of our approach as well as the benefit of future researchers. To accomplish this, we combined the solutions found during 20 independent runs of all the variants of our MaOEAs. Then we removed the dominated solutions from this collection. The resultant PFs are visualized using Figures~\ref{fig:apf_mandl4} - \ref{fig:apf_m1}. From onward, we use the abbreviations as shown in Table~\ref{tab:abbr} to refer our objective functions. 

In part(a) of these figures we present the distribution of objective values using boxplots. A boxplot summarizes a distribution using statistical descriptions. In this figure, a box spans first to last quartiles, whiskers represent 1.5x the interquartile range. We drew these boxplots using normalized values. To get an overview of the actual values, we indicated the minimum and maximum value for each objective at the two ends. In part(b), we plotted all objective vectors of the PF using a polyline, where each vertical axis represents an objective function, using normalized values. This type of figure is known as parallel coordinate. It helps to identify dependencies among objective functions. Part(c) shows the scatter matrix. It is a $7\times7$ matrix where each diagonal cell depicts the probability density of an objective function calculated using Kernel Density Estimation. Other cells depict the correlation between two objective functions.       

From all these figures, we can derive the following observations:
\begin{itemize}
	\item We can see from the boxplots that, all the distributions are skewed towards the first half for all the problem instances. This is due to the fact that all objectives functions in this study are to be minimized.
	
	\item The values of our objective functions are differently scaled. This is evident from the actual minimum and maximum values of the objectives. So we need to normalize them before deciding anything based on their relative values. 
	
	\item Considering the relative position of the means, the objective distribution of Mandl-4 is similar to Mandl-6 and the distribution of Mumford0 is similar to Mumford1.
	
	\item It is evident from the parallel coordinates that, conflicting relationship exists between several pairs of objective functions such as IVTT-WT, WT-TP, TP-UP and UP-FS for all the instances. So it is not wise to combine them using weighed sum approach which has been done by many researchers.
	
	\item Scatter matrices shows that, the nature of relationship among different objective functions is similar across all the problem instances.
	
	\item For smaller instances (Mandl-4, Mandl-6) the probability density of objective functions are unimodal. Whereas for larger instances some objective functions (i.e. IVTT, FS, DO) exhibit multimodal densities. So the risk of trapping into local optima is higher for larger instances.
\end{itemize}
%Mandl-4
\begin{figure}[!htbp]
	\centering
	\begin{subfigure}{0.6\textwidth}
		\includegraphics[width=\textwidth]{Figure/Fig/Mandl-4/RF/od}
		\caption{Boxplot}
		%\label{fig:con_pr06}
	\end{subfigure}
	\newline
	\begin{subfigure}{0.6\textwidth}
		\includegraphics[width=\textwidth]{Figure/Fig/Mandl-4/RF/pc}
		\caption{Parallel coordinate}
		%\label{fig:con_pr07}
	\end{subfigure}
	\newline
	\begin{subfigure}{0.85\textwidth}
		\includegraphics[width=\textwidth]{Figure/Fig/Mandl-4/RF/sm}
		\caption{Scatter matrix}
		%\label{fig:con_pr09}
	\end{subfigure}
	\caption{Approximated PF for Mandl (4 routes). (a) Boxplots (box spans first to last quartiles; whiskers represent 1.5x the interquartile range) show distribution of objective values. (b) Parallel coordinate visualizes each 7-dimensional objective vector as a polyline. Each vertical axis corresponds to one objective function. (c) Scatter matrix projects all objective vectors to 2-dimensional space. Each diagonal cell depicts the probability density (estimated using Kernel Density Estimation) of an objective function. Other cells depict the correlation between each pair of objective functions.}
	\label{fig:apf_mandl4}
\end{figure}
%Mandl-6
\begin{figure}[!htbp]
	\centering
	\begin{subfigure}{0.6\textwidth}
		\includegraphics[width=\textwidth]{Figure/Fig/Mandl-6/RF/od}
		\caption{Boxplot}
		%\label{fig:con_pr06}
	\end{subfigure}
	\newline
	\begin{subfigure}{0.6\textwidth}
		\includegraphics[width=\textwidth]{Figure/Fig/Mandl-6/RF/pc}
		\caption{Parallel coordinate}
		%\label{fig:con_pr07}
	\end{subfigure}
	\newline
	\begin{subfigure}{0.85\textwidth}
		\includegraphics[width=\textwidth]{Figure/Fig/Mandl-6/RF/sm}
		\caption{Scatter matrix}
		%\label{fig:con_pr09}
	\end{subfigure}
	\caption{Approximated PF for Mandl (6 routes). (a) Boxplots (box spans first to last quartiles; whiskers represent 1.5x the interquartile range) show distribution of objective values. (b) Parallel coordinate visualizes each 7-dimensional objective vector as a polyline. Each vertical axis corresponds to one objective function. (c) Scatter matrix projects all objective vectors to 2-dimensional space. Each diagonal cell depicts the probability density (estimated using Kernel Density Estimation) of an objective function. Other cells depict the correlation between each pair of objective functions.}
	\label{fig:apf_mandl6}
\end{figure}
%M0-12
\begin{figure}[!htbp]
	\centering
	\begin{subfigure}{0.6\textwidth}
		\includegraphics[width=\textwidth]{Figure/Fig/M0-12/RF/od}
		\caption{Boxplot}
		%\label{fig:con_pr06}
	\end{subfigure}
	\newline
	\begin{subfigure}{0.6\textwidth}
		\includegraphics[width=\textwidth]{Figure/Fig/M0-12/RF/pc}
		\caption{Parallel coordinate}
		%\label{fig:con_pr07}
	\end{subfigure}
	\newline
	\begin{subfigure}{0.85\textwidth}
		\includegraphics[width=\textwidth]{Figure/Fig/M0-12/RF/sm}
		\caption{Scatter matrix}
		%\label{fig:con_pr09}
	\end{subfigure}
	\caption{Approximated PF for Mumford0. (a) Boxplots (box spans first to last quartiles; whiskers represent 1.5x the interquartile range) show distribution of objective values. (b) Parallel coordinate visualizes each 7-dimensional objective vector as a polyline. Each vertical axis corresponds to one objective function. (c) Scatter matrix projects all objective vectors to 2-dimensional space. Each diagonal cell depicts the probability density (estimated using Kernel Density Estimation) of an objective function. Other cells depict the correlation between each pair of objective functions.}
	\label{fig:apf_m0}
\end{figure}
%M1-15
\begin{figure}[!htbp]
	\centering
	\begin{subfigure}{0.6\textwidth}
		\includegraphics[width=\textwidth]{Figure/Fig/M1-15/RF/od}
		\caption{Boxplot}
		%\label{fig:con_pr06}
	\end{subfigure}
	\newline
	\begin{subfigure}{0.6\textwidth}
		\includegraphics[width=\textwidth]{Figure/Fig/M1-15/RF/pc}
		\caption{Parallel coordinate}
		%\label{fig:con_pr07}
	\end{subfigure}
	\newline
	\begin{subfigure}{0.85\textwidth}
		\includegraphics[width=\textwidth]{Figure/Fig/M1-15/RF/sm}
		\caption{Scatter matrix}
		%\label{fig:con_pr09}
	\end{subfigure}
	\caption{Approximated PF for Mumford1. (a) Boxplots (box spans first to last quartiles; whiskers represent 1.5x the interquartile range) show distribution of objective values. (b) Parallel coordinate visualizes each 7-dimensional objective vector as a polyline. Each vertical axis corresponds to one objective function. (c) Scatter matrix projects all objective vectors to 2-dimensional space. Each diagonal cell depicts the probability density (estimated using Kernel Density Estimation) of an objective function. Other cells depict the correlation between each pair of objective functions.}
	\label{fig:apf_m1}
\end{figure}

\subsection{Effect of Crossover}

In this study we adapted four MaOEAs, MOEA/D, SPEA2, NSGAIII and $\theta$-DEA, for solving TNDP. To find the effect of our crossover operator on MaOEAs' performance, we conducted experiments with different crossover rates as shown in Table~\ref{tab:parameters}. We summarize the HV results of 20 independent runs for each possible combination (problem instance, algorithm, mutation scheme) using a boxplot shown in Figures ~\ref{fig:cr_mandl4_moead} - \ref{fig:cr_m1_thetadea}. 

Figures~\ref{fig:cr_mandl4_moead} - \ref{fig:cr_mandl4_thetadea} present the HV results for Mandl-4. Here we see that crossover improves the performance of all MaOEAs except for MOEA/D. For SPEA2, the increase of crossover rate consistently improves the performance when Basic Scheme and Guided Scheme are used as mutation scheme. However the effect is uneven for other schemes. Regarding NSGAIII, increasing crossover rate results consistent increase in performance for Random Scheme I and Guided Scheme. For other scheme we do not find any obvious trend. In case of $\theta$-DEA, the performance is not improved continuously for any mutation scheme. But here we see a pattern for all the schemes which is, the performance gradually improves until the crossover rate reaches around 0.8.

We summarize the  HV results for Mandl-6 in Figures~\ref{fig:cr_mandl6_moead} - \ref{fig:cr_mandl6_thetadea}. Again we see that crossover improves algorithm performance other than MOEA/D. These figures bear almost the same information as with Mandl-4. However a difference can be noted from careful examination. In case of SPEA2 and NSGAIII, the extent of uneven change in performance is somewhat reduced and it seems that the improvement converges around crossover rate = 0.8.   

Now we move to larger datasets. At first we show the HV results for Mumford0 using Figures~\ref{fig:cr_m0_moead} - \ref{fig:cr_m0_thetadea}. Here we find notable changes in the  behavior of our MaOEAs. We observe that now crossover can improve NSGAIII and $\theta$-DEA. For NSGAIII, the improvement occurs for Random Scheme III and Guided Scheme where only the later scheme shows consistent improvement.  In case of $\theta$-DEA, all the mutation schemes exhibit consistent improvement. And unlike previous cases, here we see a sharp rise in HV after crossover rate exceeds 0.8.

Finally in Figures~\ref{fig:cr_m1_moead} - \ref{fig:cr_m1_thetadea} we summarize the HV results for Mumford1 which is the largest dataset . And we see that only $\theta$-DEA benefits from crossover. With all the mutation schemes it exhibit similar trend as with Mumford0. The rest MaOEAs fail to improve with crossover.

We notice that crossover failed to bring any improvement in case of MOEA/D. We can explain this phenomenon using the algorithm properties. In most of the time, MOEA/D selects two neighboring solutions from the population as parents for crossover. These solutions may represent similar transit network. Moreover, crossover generates an offspring which also may be similar to its parents. The offspring will replace its current neighboring solutions if it is better than them. As the neighborhood is defined based on uniformly distributed weight vectors, it is very likely that a high quality offspring will gradually fill the large portion of the population with its variants. As a results, diversity among the population members will be reduced significantly. That's why, the use of crossover operator in MOEA/D results deteriorations in HV values. 

Following conclusions can be made from the results presented in this section:
\begin{itemize}
	\item The crossover operator cannot improve the performance of MOEA/D at all.
	
	\item For SPEA2 and NSGAIII, crossover has positive effect only for smaller instances (Mandl-4, Mandl-6).
	
	\item Crossover improves the performance of $\theta$-DEA for all the instances. Higher crossover rate causes a sharp rise in performance for larger instances (Mumford0, Mumford1).
	
	\item For a particular problem instance, the crossover rate should be carefully chosen based on the type of mutation scheme.
\end{itemize}

%Mandl-4
\begin{figure}[!htbp]
	\centering
	\begin{subfigure}[b]{0.52\textwidth}
		\includegraphics[width=\textwidth]{Figure/Fig/Mandl-4/MOEAD/basic_scheme.png}
		%\caption{pr03}
		%\label{fig:con_pr03}
	\end{subfigure}%
	\begin{subfigure}[b]{0.52\textwidth}
		\includegraphics[width=\textwidth]{Figure/Fig/Mandl-4/MOEAD/random_scheme_i.png}
		%\caption{pr06}
		%\label{fig:con_pr06}
	\end{subfigure}%
	\newline
	\begin{subfigure}[b]{0.52\textwidth}
		\includegraphics[width=\textwidth]{Figure/Fig/Mandl-4/MOEAD/random_scheme_ii.png}
		%\caption{pr07}
		%\label{fig:con_pr07}
	\end{subfigure}%
	\begin{subfigure}[b]{0.52\textwidth}
		\includegraphics[width=\textwidth]{Figure/Fig/Mandl-4/MOEAD/random_scheme_iii.png}
		%\caption{pr08}
		%\label{fig:con_pr08}
	\end{subfigure}%
	\newline
	\begin{subfigure}[b]{0.52\textwidth}
		\includegraphics[width=\textwidth]{Figure/Fig/Mandl-4/MOEAD/guided_scheme.png}
		%\caption{pr09}
		%\label{fig:con_pr09}
	\end{subfigure}%
	\caption{Effect of crossover rate on MOEA/D for Mandl (4 routes). Each figure corresponds to each mutation scheme. In each figure there are eight boxplots for eight crossover rates.  Each boxplot (box spans first to last quartiles; whiskers represent 1.5x the interquartile range) summarizes the HV results of 20 independent runs.}
	\label{fig:cr_mandl4_moead}
\end{figure}
\begin{figure}[!htbp]
	\centering
	\begin{subfigure}[b]{0.52\textwidth}
		\includegraphics[width=\textwidth]{Figure/Fig/Mandl-4/SPEA2/basic_scheme.png}
		%\caption{pr03}
		%\label{fig:con_pr03}
	\end{subfigure}%
	\begin{subfigure}[b]{0.52\textwidth}
		\includegraphics[width=\textwidth]{Figure/Fig/Mandl-4/SPEA2/random_scheme_i.png}
		%\caption{pr06}
		%\label{fig:con_pr06}
	\end{subfigure}%
	\newline
	\begin{subfigure}[b]{0.52\textwidth}
		\includegraphics[width=\textwidth]{Figure/Fig/Mandl-4/SPEA2/random_scheme_ii.png}
		%\caption{pr07}
		%\label{fig:con_pr07}
	\end{subfigure}%
	\begin{subfigure}[b]{0.52\textwidth}
		\includegraphics[width=\textwidth]{Figure/Fig/Mandl-4/SPEA2/random_scheme_iii.png}
		%\caption{pr08}
		%\label{fig:con_pr08}
	\end{subfigure}%
	\newline
	\begin{subfigure}[b]{0.52\textwidth}
		\includegraphics[width=\textwidth]{Figure/Fig/Mandl-4/SPEA2/guided_scheme.png}
		%\caption{pr09}
		%\label{fig:con_pr09}
	\end{subfigure}%
	\caption{Effect of crossover rate on SPEA2 for Mandl (4 routes). Each figure corresponds to each mutation scheme. In each figure there are eight boxplots for eight crossover rates.  Each boxplot (box spans first to last quartiles; whiskers represent 1.5x the interquartile range) summarizes the HV results of 20 independent runs.}
	\label{fig:cr_mandl4_spea2}
\end{figure}
\begin{figure}[!htbp]
	\centering
	\begin{subfigure}[b]{0.52\textwidth}
		\includegraphics[width=\textwidth]{Figure/Fig/Mandl-4/NSGAIII/basic_scheme.png}
		%\caption{pr03}
		%\label{fig:con_pr03}
	\end{subfigure}%
	\begin{subfigure}[b]{0.52\textwidth}
		\includegraphics[width=\textwidth]{Figure/Fig/Mandl-4/NSGAIII/random_scheme_i.png}
		%\caption{pr06}
		%\label{fig:con_pr06}
	\end{subfigure}%
	\newline
	\begin{subfigure}[b]{0.52\textwidth}
		\includegraphics[width=\textwidth]{Figure/Fig/Mandl-4/NSGAIII/random_scheme_ii.png}
		%\caption{pr07}
		%\label{fig:con_pr07}
	\end{subfigure}%
	\begin{subfigure}[b]{0.52\textwidth}
		\includegraphics[width=\textwidth]{Figure/Fig/Mandl-4/NSGAIII/random_scheme_iii.png}
		%\caption{pr08}
		%\label{fig:con_pr08}
	\end{subfigure}%
	\newline
	\begin{subfigure}[b]{0.52\textwidth}
		\includegraphics[width=\textwidth]{Figure/Fig/Mandl-4/NSGAIII/guided_scheme.png}
		%\caption{pr09}
		%\label{fig:con_pr09}
	\end{subfigure}%
	\caption{Effect of crossover rate on NSGAIII for Mandl (4 routes). Each figure corresponds to each mutation scheme. In each figure there are eight boxplots for eight crossover rates.  Each boxplot (box spans first to last quartiles; whiskers represent 1.5x the interquartile range) summarizes the HV results of 20 independent runs.}
	\label{fig:cr_mandl4_nsgaiii}
\end{figure}
\begin{figure}[!htbp]
	\centering
	\begin{subfigure}[b]{0.52\textwidth}
		\includegraphics[width=\textwidth]{Figure/Fig/Mandl-4/ThetaDEA/basic_scheme.png}
		%\caption{pr03}
		%\label{fig:con_pr03}
	\end{subfigure}%
	\begin{subfigure}[b]{0.52\textwidth}
		\includegraphics[width=\textwidth]{Figure/Fig/Mandl-4/ThetaDEA/random_scheme_i.png}
		%\caption{pr06}
		%\label{fig:con_pr06}
	\end{subfigure}%
	\newline
	\begin{subfigure}[b]{0.52\textwidth}
		\includegraphics[width=\textwidth]{Figure/Fig/Mandl-4/ThetaDEA/random_scheme_ii.png}
		%\caption{pr07}
		%\label{fig:con_pr07}
	\end{subfigure}%
	\begin{subfigure}[b]{0.52\textwidth}
		\includegraphics[width=\textwidth]{Figure/Fig/Mandl-4/ThetaDEA/random_scheme_iii.png}
		%\caption{pr08}
		%\label{fig:con_pr08}
	\end{subfigure}%
	\newline
	\begin{subfigure}[b]{0.52\textwidth}
		\includegraphics[width=\textwidth]{Figure/Fig/Mandl-4/ThetaDEA/guided_scheme.png}
		%\caption{pr09}
		%\label{fig:con_pr09}
	\end{subfigure}%
	\caption{Effect of crossover rate on $\theta$-DEA for Mandl (4 routes). Each figure corresponds to each mutation scheme. In each figure there are eight boxplots for eight crossover rates.  Each boxplot (box spans first to last quartiles; whiskers represent 1.5x the interquartile range) summarizes the HV results of 20 independent runs.}
	\label{fig:cr_mandl4_thetadea}
\end{figure}

%Mandl-6
\begin{figure}[!htbp]
	\centering
	\begin{subfigure}[b]{0.52\textwidth}
		\includegraphics[width=\textwidth]{Figure/Fig/Mandl-6/MOEAD/basic_scheme.png}
		%\caption{pr03}
		%\label{fig:con_pr03}
	\end{subfigure}%
	\begin{subfigure}[b]{0.52\textwidth}
		\includegraphics[width=\textwidth]{Figure/Fig/Mandl-6/MOEAD/random_scheme_i.png}
		%\caption{pr06}
		%\label{fig:con_pr06}
	\end{subfigure}%
	\newline
	\begin{subfigure}[b]{0.52\textwidth}
		\includegraphics[width=\textwidth]{Figure/Fig/Mandl-6/MOEAD/random_scheme_ii.png}
		%\caption{pr07}
		%\label{fig:con_pr07}
	\end{subfigure}%
	\begin{subfigure}[b]{0.52\textwidth}
		\includegraphics[width=\textwidth]{Figure/Fig/Mandl-6/MOEAD/random_scheme_iii.png}
		%\caption{pr08}
		%\label{fig:con_pr08}
	\end{subfigure}%
	\newline
	\begin{subfigure}[b]{0.52\textwidth}
		\includegraphics[width=\textwidth]{Figure/Fig/Mandl-6/MOEAD/guided_scheme.png}
		%\caption{pr09}
		%\label{fig:con_pr09}
	\end{subfigure}%
	\caption{Effect of crossover rate on MOEA/D for Mandl (6 routes). Each figure corresponds to each mutation scheme. In each figure there are eight boxplots for eight crossover rates.  Each boxplot (box spans first to last quartiles; whiskers represent 1.5x the interquartile range) summarizes the HV results of 20 independent runs.}
	\label{fig:cr_mandl6_moead}
\end{figure}
\begin{figure}[!htbp]
	\centering
	\begin{subfigure}[b]{0.52\textwidth}
		\includegraphics[width=\textwidth]{Figure/Fig/Mandl-6/SPEA2/basic_scheme.png}
		%\caption{pr03}
		%\label{fig:con_pr03}
	\end{subfigure}%
	\begin{subfigure}[b]{0.52\textwidth}
		\includegraphics[width=\textwidth]{Figure/Fig/Mandl-6/SPEA2/random_scheme_i.png}
		%\caption{pr06}
		%\label{fig:con_pr06}
	\end{subfigure}%
	\newline
	\begin{subfigure}[b]{0.52\textwidth}
		\includegraphics[width=\textwidth]{Figure/Fig/Mandl-6/SPEA2/random_scheme_ii.png}
		%\caption{pr07}
		%\label{fig:con_pr07}
	\end{subfigure}%
	\begin{subfigure}[b]{0.52\textwidth}
		\includegraphics[width=\textwidth]{Figure/Fig/Mandl-6/SPEA2/random_scheme_iii.png}
		%\caption{pr08}
		%\label{fig:con_pr08}
	\end{subfigure}%
	\newline
	\begin{subfigure}[b]{0.52\textwidth}
		\includegraphics[width=\textwidth]{Figure/Fig/Mandl-6/SPEA2/guided_scheme.png}
		%\caption{pr09}
		%\label{fig:con_pr09}
	\end{subfigure}%
	\caption{Effect of crossover rate on SPEA2 for Mandl (6 routes). Each figure corresponds to each mutation scheme. In each figure there are eight boxplots for eight crossover rates.  Each boxplot (box spans first to last quartiles; whiskers represent 1.5x the interquartile range) summarizes the HV results of 20 independent runs.}
	\label{fig:cr_mandl6_spea2}
\end{figure}
\begin{figure}[!htbp]
	\centering
	\begin{subfigure}[b]{0.52\textwidth}
		\includegraphics[width=\textwidth]{Figure/Fig/Mandl-6/NSGAIII/basic_scheme.png}
		%\caption{pr03}
		%\label{fig:con_pr03}
	\end{subfigure}%
	\begin{subfigure}[b]{0.52\textwidth}
		\includegraphics[width=\textwidth]{Figure/Fig/Mandl-6/NSGAIII/random_scheme_i.png}
		%\caption{pr06}
		%\label{fig:con_pr06}
	\end{subfigure}%
	\newline
	\begin{subfigure}[b]{0.52\textwidth}
		\includegraphics[width=\textwidth]{Figure/Fig/Mandl-6/NSGAIII/random_scheme_ii.png}
		%\caption{pr07}
		%\label{fig:con_pr07}
	\end{subfigure}%
	\begin{subfigure}[b]{0.52\textwidth}
		\includegraphics[width=\textwidth]{Figure/Fig/Mandl-6/NSGAIII/random_scheme_iii.png}
		%\caption{pr08}
		%\label{fig:con_pr08}
	\end{subfigure}%
	\newline
	\begin{subfigure}[b]{0.52\textwidth}
		\includegraphics[width=\textwidth]{Figure/Fig/Mandl-6/NSGAIII/guided_scheme.png}
		%\caption{pr09}
		%\label{fig:con_pr09}
	\end{subfigure}%
	\caption{Effect of crossover rate on NSGAIII for Mandl (6 routes). Each figure corresponds to each mutation scheme. In each figure there are eight boxplots for eight crossover rates.  Each boxplot (box spans first to last quartiles; whiskers represent 1.5x the interquartile range) summarizes the HV results of 20 independent runs.}
	\label{fig:cr_mandl6_nsgaiii}
\end{figure}
\begin{figure}[!htbp]
	\centering
	\begin{subfigure}[b]{0.52\textwidth}
		\includegraphics[width=\textwidth]{Figure/Fig/Mandl-6/ThetaDEA/basic_scheme.png}
		%\caption{pr03}
		%\label{fig:con_pr03}
	\end{subfigure}%
	\begin{subfigure}[b]{0.52\textwidth}
		\includegraphics[width=\textwidth]{Figure/Fig/Mandl-6/ThetaDEA/random_scheme_i.png}
		%\caption{pr06}
		%\label{fig:con_pr06}
	\end{subfigure}%
	\newline
	\begin{subfigure}[b]{0.52\textwidth}
		\includegraphics[width=\textwidth]{Figure/Fig/Mandl-6/ThetaDEA/random_scheme_ii.png}
		%\caption{pr07}
		%\label{fig:con_pr07}
	\end{subfigure}%
	\begin{subfigure}[b]{0.52\textwidth}
		\includegraphics[width=\textwidth]{Figure/Fig/Mandl-6/ThetaDEA/random_scheme_iii.png}
		%\caption{pr08}
		%\label{fig:con_pr08}
	\end{subfigure}%
	\newline
	\begin{subfigure}[b]{0.52\textwidth}
		\includegraphics[width=\textwidth]{Figure/Fig/Mandl-6/ThetaDEA/guided_scheme.png}
		%\caption{pr09}
		%\label{fig:con_pr09}
	\end{subfigure}%
	\caption{Effect of crossover rate on $\theta$-DEA for Mandl (6 routes). Each figure corresponds to each mutation scheme. In each figure there are eight boxplots for eight crossover rates.  Each boxplot (box spans first to last quartiles; whiskers represent 1.5x the interquartile range) summarizes the HV results of 20 independent runs.}
	\label{fig:cr_mandl6_thetadea}
\end{figure}

%Mumford0
\begin{figure}[!htbp]
	\centering
	\begin{subfigure}[b]{0.52\textwidth}
		\includegraphics[width=\textwidth]{Figure/Fig/M0-12/MOEAD/basic_scheme.png}
		%\caption{pr03}
		%\label{fig:con_pr03}
	\end{subfigure}%
	\begin{subfigure}[b]{0.52\textwidth}
		\includegraphics[width=\textwidth]{Figure/Fig/M0-12/MOEAD/random_scheme_i.png}
		%\caption{pr06}
		%\label{fig:con_pr06}
	\end{subfigure}%
	\newline
	\begin{subfigure}[b]{0.52\textwidth}
		\includegraphics[width=\textwidth]{Figure/Fig/M0-12/MOEAD/random_scheme_ii.png}
		%\caption{pr07}
		%\label{fig:con_pr07}
	\end{subfigure}%
	\begin{subfigure}[b]{0.52\textwidth}
		\includegraphics[width=\textwidth]{Figure/Fig/M0-12/MOEAD/random_scheme_iii.png}
		%\caption{pr08}
		%\label{fig:con_pr08}
	\end{subfigure}%
	\newline
	\begin{subfigure}[b]{0.52\textwidth}
		\includegraphics[width=\textwidth]{Figure/Fig/M0-12/MOEAD/guided_scheme.png}
		%\caption{pr09}
		%\label{fig:con_pr09}
	\end{subfigure}%
	\caption{Effect of crossover rate on MOEA/D for Mumford0. Each figure corresponds to each mutation scheme. In each figure there are eight boxplots for eight crossover rates.  Each boxplot (box spans first to last quartiles; whiskers represent 1.5x the interquartile range) summarizes the HV results of 20 independent runs.}
	\label{fig:cr_m0_moead}
\end{figure}
\begin{figure}[!htbp]
	\centering
	\begin{subfigure}[b]{0.52\textwidth}
		\includegraphics[width=\textwidth]{Figure/Fig/M0-12/SPEA2/basic_scheme.png}
		%\caption{pr03}
		%\label{fig:con_pr03}
	\end{subfigure}%
	\begin{subfigure}[b]{0.52\textwidth}
		\includegraphics[width=\textwidth]{Figure/Fig/M0-12/SPEA2/random_scheme_i.png}
		%\caption{pr06}
		%\label{fig:con_pr06}
	\end{subfigure}%
	\newline
	\begin{subfigure}[b]{0.52\textwidth}
		\includegraphics[width=\textwidth]{Figure/Fig/M0-12/SPEA2/random_scheme_ii.png}
		%\caption{pr07}
		%\label{fig:con_pr07}
	\end{subfigure}%
	\begin{subfigure}[b]{0.52\textwidth}
		\includegraphics[width=\textwidth]{Figure/Fig/M0-12/SPEA2/random_scheme_iii.png}
		%\caption{pr08}
		%\label{fig:con_pr08}
	\end{subfigure}%
	\newline
	\begin{subfigure}[b]{0.52\textwidth}
		\includegraphics[width=\textwidth]{Figure/Fig/M0-12/SPEA2/guided_scheme.png}
		%\caption{pr09}
		%\label{fig:con_pr09}
	\end{subfigure}%
	\caption{Effect of crossover rate on SPEA2 for Mumford0. Each figure corresponds to each mutation scheme. In each figure there are eight boxplots for eight crossover rates.  Each boxplot (box spans first to last quartiles; whiskers represent 1.5x the interquartile range) summarizes the HV results of 20 independent runs.}
	\label{fig:cr_m0_spea2}
\end{figure}
\begin{figure}[!htbp]
	\centering
	\begin{subfigure}[b]{0.52\textwidth}
		\includegraphics[width=\textwidth]{Figure/Fig/M0-12/NSGAIII/basic_scheme.png}
		%\caption{pr03}
		%\label{fig:con_pr03}
	\end{subfigure}%
	\begin{subfigure}[b]{0.52\textwidth}
		\includegraphics[width=\textwidth]{Figure/Fig/M0-12/NSGAIII/random_scheme_i.png}
		%\caption{pr06}
		%\label{fig:con_pr06}
	\end{subfigure}%
	\newline
	\begin{subfigure}[b]{0.52\textwidth}
		\includegraphics[width=\textwidth]{Figure/Fig/M0-12/NSGAIII/random_scheme_ii.png}
		%\caption{pr07}
		%\label{fig:con_pr07}
	\end{subfigure}%
	\begin{subfigure}[b]{0.52\textwidth}
		\includegraphics[width=\textwidth]{Figure/Fig/M0-12/NSGAIII/random_scheme_iii.png}
		%\caption{pr08}
		%\label{fig:con_pr08}
	\end{subfigure}%
	\newline
	\begin{subfigure}[b]{0.52\textwidth}
		\includegraphics[width=\textwidth]{Figure/Fig/M0-12/NSGAIII/guided_scheme.png}
		%\caption{pr09}
		%\label{fig:con_pr09}
	\end{subfigure}%
	\caption{Effect of crossover rate on NSGAIII for Mumford0. Each figure corresponds to each mutation scheme. In each figure there are eight boxplots for eight crossover rates.  Each boxplot (box spans first to last quartiles; whiskers represent 1.5x the interquartile range) summarizes the HV results of 20 independent runs.}
	\label{fig:cr_m0_nsgaiii}
\end{figure}
\begin{figure}[!htbp]
	\centering
	\begin{subfigure}[b]{0.52\textwidth}
		\includegraphics[width=\textwidth]{Figure/Fig/M0-12/ThetaDEA/basic_scheme.png}
		%\caption{pr03}
		%\label{fig:con_pr03}
	\end{subfigure}%
	\begin{subfigure}[b]{0.52\textwidth}
		\includegraphics[width=\textwidth]{Figure/Fig/M0-12/ThetaDEA/random_scheme_i.png}
		%\caption{pr06}
		%\label{fig:con_pr06}
	\end{subfigure}%
	\newline
	\begin{subfigure}[b]{0.52\textwidth}
		\includegraphics[width=\textwidth]{Figure/Fig/M0-12/ThetaDEA/random_scheme_ii.png}
		%\caption{pr07}
		%\label{fig:con_pr07}
	\end{subfigure}%
	\begin{subfigure}[b]{0.52\textwidth}
		\includegraphics[width=\textwidth]{Figure/Fig/M0-12/ThetaDEA/random_scheme_iii.png}
		%\caption{pr08}
		%\label{fig:con_pr08}
	\end{subfigure}%
	\newline
	\begin{subfigure}[b]{0.52\textwidth}
		\includegraphics[width=\textwidth]{Figure/Fig/M0-12/ThetaDEA/guided_scheme.png}
		%\caption{pr09}
		%\label{fig:con_pr09}
	\end{subfigure}%
	\caption{Effect of crossover rate on $\theta$-DEA for Mumford0. Each figure corresponds to each mutation scheme. In each figure there are eight boxplots for eight crossover rates.  Each boxplot (box spans first to last quartiles; whiskers represent 1.5x the interquartile range) summarizes the HV results of 20 independent runs.}
	\label{fig:cr_m0_thetadea}
\end{figure}

%Mumford1
\begin{figure}[!htbp]
	\centering
	\begin{subfigure}[b]{0.52\textwidth}
		\includegraphics[width=\textwidth]{Figure/Fig/M1-15/MOEAD/basic_scheme.png}
		%\caption{pr03}
		%\label{fig:con_pr03}
	\end{subfigure}%
	\begin{subfigure}[b]{0.52\textwidth}
		\includegraphics[width=\textwidth]{Figure/Fig/M1-15/MOEAD/random_scheme_i.png}
		%\caption{pr06}
		%\label{fig:con_pr06}
	\end{subfigure}%
	\newline
	\begin{subfigure}[b]{0.52\textwidth}
		\includegraphics[width=\textwidth]{Figure/Fig/M1-15/MOEAD/random_scheme_ii.png}
		%\caption{pr07}
		%\label{fig:con_pr07}
	\end{subfigure}%
	\begin{subfigure}[b]{0.52\textwidth}
		\includegraphics[width=\textwidth]{Figure/Fig/M1-15/MOEAD/random_scheme_iii.png}
		%\caption{pr08}
		%\label{fig:con_pr08}
	\end{subfigure}%
	\newline
	\begin{subfigure}[b]{0.52\textwidth}
		\includegraphics[width=\textwidth]{Figure/Fig/M1-15/MOEAD/guided_scheme.png}
		%\caption{pr09}
		%\label{fig:con_pr09}
	\end{subfigure}%
	\caption{Effect of crossover rate on MOEA/D for Mumford1. Each figure corresponds to each mutation scheme. In each figure there are eight boxplots for eight crossover rates.  Each boxplot (box spans first to last quartiles; whiskers represent 1.5x the interquartile range) summarizes the HV results of 20 independent runs.}
	\label{fig:cr_m1_moead}
\end{figure}
\begin{figure}[!htbp]
	\centering
	\begin{subfigure}[b]{0.52\textwidth}
		\includegraphics[width=\textwidth]{Figure/Fig/M1-15/SPEA2/basic_scheme.png}
		%\caption{pr03}
		%\label{fig:con_pr03}
	\end{subfigure}%
	\begin{subfigure}[b]{0.52\textwidth}
		\includegraphics[width=\textwidth]{Figure/Fig/M1-15/SPEA2/random_scheme_i.png}
		%\caption{pr06}
		%\label{fig:con_pr06}
	\end{subfigure}%
	\newline
	\begin{subfigure}[b]{0.52\textwidth}
		\includegraphics[width=\textwidth]{Figure/Fig/M1-15/SPEA2/random_scheme_ii.png}
		%\caption{pr07}
		%\label{fig:con_pr07}
	\end{subfigure}%
	\begin{subfigure}[b]{0.52\textwidth}
		\includegraphics[width=\textwidth]{Figure/Fig/M1-15/SPEA2/random_scheme_iii.png}
		%\caption{pr08}
		%\label{fig:con_pr08}
	\end{subfigure}%
	\newline
	\begin{subfigure}[b]{0.52\textwidth}
		\includegraphics[width=\textwidth]{Figure/Fig/M1-15/SPEA2/guided_scheme.png}
		%\caption{pr09}
		%\label{fig:con_pr09}
	\end{subfigure}%
	\caption{Effect of crossover rate on SPEA2 for Mumford1. Each figure corresponds to each mutation scheme. In each figure there are eight boxplots for eight crossover rates.  Each boxplot (box spans first to last quartiles; whiskers represent 1.5x the interquartile range) summarizes the HV results of 20 independent runs.}
	\label{fig:cr_m1_spea2}
\end{figure}
\begin{figure}[!htbp]
	\centering
	\begin{subfigure}[b]{0.52\textwidth}
		\includegraphics[width=\textwidth]{Figure/Fig/M1-15/NSGAIII/basic_scheme.png}
		%\caption{pr03}
		%\label{fig:con_pr03}
	\end{subfigure}%
	\begin{subfigure}[b]{0.52\textwidth}
		\includegraphics[width=\textwidth]{Figure/Fig/M1-15/NSGAIII/random_scheme_i.png}
		%\caption{pr06}
		%\label{fig:con_pr06}
	\end{subfigure}%
	\newline
	\begin{subfigure}[b]{0.52\textwidth}
		\includegraphics[width=\textwidth]{Figure/Fig/M1-15/NSGAIII/random_scheme_ii.png}
		%\caption{pr07}
		%\label{fig:con_pr07}
	\end{subfigure}%
	\begin{subfigure}[b]{0.52\textwidth}
		\includegraphics[width=\textwidth]{Figure/Fig/M1-15/NSGAIII/random_scheme_iii.png}
		%\caption{pr08}
		%\label{fig:con_pr08}
	\end{subfigure}%
	\newline
	\begin{subfigure}[b]{0.52\textwidth}
		\includegraphics[width=\textwidth]{Figure/Fig/M1-15/NSGAIII/guided_scheme.png}
		%\caption{pr09}
		%\label{fig:con_pr09}
	\end{subfigure}%
	\caption{Effect of crossover rate on NSGAIII for Mumford1. Each figure corresponds to each mutation scheme. In each figure there are eight boxplots for eight crossover rates.  Each boxplot (box spans first to last quartiles; whiskers represent 1.5x the interquartile range) summarizes the HV results of 20 independent runs.}
	\label{fig:cr_m1_nsgaiii}
\end{figure}
\begin{figure}[!htbp]
	\centering
	\begin{subfigure}[b]{0.52\textwidth}
		\includegraphics[width=\textwidth]{Figure/Fig/M1-15/ThetaDEA/basic_scheme.png}
		%\caption{pr03}
		%\label{fig:con_pr03}
	\end{subfigure}%
	\begin{subfigure}[b]{0.52\textwidth}
		\includegraphics[width=\textwidth]{Figure/Fig/M1-15/ThetaDEA/random_scheme_i.png}
		%\caption{pr06}
		%\label{fig:con_pr06}
	\end{subfigure}%
	\newline
	\begin{subfigure}[b]{0.52\textwidth}
		\includegraphics[width=\textwidth]{Figure/Fig/M1-15/ThetaDEA/random_scheme_ii.png}
		%\caption{pr07}
		%\label{fig:con_pr07}
	\end{subfigure}%
	\begin{subfigure}[b]{0.52\textwidth}
		\includegraphics[width=\textwidth]{Figure/Fig/M1-15/ThetaDEA/random_scheme_iii.png}
		%\caption{pr08}
		%\label{fig:con_pr08}
	\end{subfigure}%
	\newline
	\begin{subfigure}[b]{0.52\textwidth}
		\includegraphics[width=\textwidth]{Figure/Fig/M1-15/ThetaDEA/guided_scheme.png}
		%\caption{pr09}
		%\label{fig:con_pr09}
	\end{subfigure}%
	\caption{Effect of crossover rate on $\theta$-DEA for Mumford1. Each figure corresponds to each mutation scheme. In each figure there are eight boxplots for eight crossover rates.  Each boxplot (box spans first to last quartiles; whiskers represent 1.5x the interquartile range) summarizes the HV results of 20 independent runs.}
	\label{fig:cr_m1_thetadea}
\end{figure}

\subsection{Effect of Mutation Scheme}
We devised four mutation schemes, Basic Scheme, Random Scheme I, Random Scheme II, Random Scheme III and Guided Scheme, to work with MaOEAs. Now we analyze their effect on MaOEAs' performance. To accomplish this, for each MaOEA we created mutation scheme wise front by combining all non-dominated solutions generated using a particular mutation scheme. Then we calculated HV for each resulting front. In this way we get a HV value for each possible combination (problem instance, algorithm, mutation scheme). These values are shown in Table~\ref{tab:mutwise_hv}. 

% Table generated by Excel2LaTeX from sheet 'MutWiseHV'
\begin{table}[!htbp]
	\centering
	\caption{Mutation scheme wise HV values.}
	\begin{tabular}{|c|r|r|r|r|r|}
		\hline
		\multirow{2}[4]{*}{Instance} & \multicolumn{1}{c|}{\multirow{2}[4]{*}{Mutation scheme}} & \multicolumn{4}{c|}{HV} \\
		\cline{3-6}          & \multicolumn{1}{c|}{} & \multicolumn{1}{c|}{SPEA2} & \multicolumn{1}{c|}{MOEA/D} & \multicolumn{1}{c|}{NSGAIII} & \multicolumn{1}{c|}{$\theta$-DEA} \\
		\hline
		\multirow{5}[10]{*}{Mandl-4} & Basic Scheme & 0.5675 & 0.4311 & 0.5677 & 0.5609 \\
		\cline{2-6}          & Random Scheme I & 0.5589 & 0.4224 & 0.5666 & 0.5672 \\
		\cline{2-6}          & Random Scheme II & 0.5643 & 0.4487 & 0.5680 & 0.5707 \\
		\cline{2-6}          & Random Scheme III & 0.5573 & 0.4398 & 0.5636 & 0.5678 \\
		\cline{2-6}          & Guided Scheme & 0.5663 & 0.4458 & 0.5707 & 0.5583 \\
		\hline
		\multirow{5}[10]{*}{Mandl-6} & Basic Scheme & 0.5890 & 0.4598 & 0.6032 & 0.5982 \\
		\cline{2-6}          & Random Scheme I & 0.5919 & 0.4415 & 0.6045 & 0.5924 \\
		\cline{2-6}          & Random Scheme II & 0.5922 & 0.4572 & 0.6048 & 0.6032 \\
		\cline{2-6}          & Random Scheme III & 0.5909 & 0.4327 & 0.6014 & 0.6007 \\
		\cline{2-6}          & Guided Scheme & 0.5904 & 0.4340 & 0.6041 & 0.5973 \\
		\hline
		\multirow{5}[10]{*}{Mumford0} & Basic Scheme & 0.3598 & 0.2495 & 0.4002 & 0.4251 \\
		\cline{2-6}          & Random Scheme I & 0.3459 & 0.2503 & 0.3786 & 0.4108 \\
		\cline{2-6}          & Random Scheme II & 0.3596 & 0.2610 & 0.3973 & 0.4136 \\
		\cline{2-6}          & Random Scheme III & 0.3455 & 0.2652 & 0.3791 & 0.4023 \\
		\cline{2-6}          & Guided Scheme & 0.3618 & 0.2397 & 0.3958 & 0.4215 \\
		\hline
		\multirow{5}[10]{*}{Mumford1} & Basic Scheme & 0.1826 & 0.1820 & 0.2152 & 0.2365 \\
		\cline{2-6}          & Random Scheme I & 0.1634 & 0.1827 & 0.2012 & 0.2299 \\
		\cline{2-6}          & Random Scheme II & 0.1879 & 0.1940 & 0.2243 & 0.2360 \\
		\cline{2-6}          & Random Scheme III & 0.1719 & 0.1879 & 0.2122 & 0.2346 \\
		\cline{2-6}          & Guided Scheme & 0.1808 & 0.1866 & 0.2245 & 0.2457 \\
		\hline
	\end{tabular}%
	\label{tab:mutwise_hv}%
\end{table}%


Figure~\ref{fig:mandl4_mutwise_hv} depicts mutation scheme wise HV results for Mandl-4 using a bar chart. We see that the relative performance of the mutation schemes varies across the MaOEAs. For SPEA2, Basic Scheme achieves the best performance. Within NSGAIII, Guided Scheme defeats others. And for MOEA/D and $\theta$-DEA, Random Scheme II wins. Among MaOEAs, NSGAIII and $\theta$-DEA show the best results while MOEA/D shows the worst.  

\begin{figure} [!htbp]
	\centering
	\includegraphics[width=16cm]{Figure/Fig/Mandl-4/mutwise_hv.eps}
	\caption{Effect of mutation schemes on different algorithms for Mandl (4 routes). Each vertical bar represents the performance of a mutation scheme for an algorithm using HV value. For each algorithm we used five mutation schemes. So five vertical bars are grouped under each algorithm.}\label{fig:mandl4_mutwise_hv} 
\end{figure}

We present mutation scheme wise HV results for Mandl-6 in Figure~\ref{fig:mandl6_mutwise_hv}. Here relative position of the mutation schemes within a MaOEA differs from Mandl-4. Basic Scheme achieves the best performance in case of MOEA/D. And for SPEA2, NSGAIII and $\theta$-DEA, Random Scheme II wins. Again, NSGAIII and $\theta$-DEA show the best results while MOEA/D shows the worst.  

\begin{figure} [!htbp]
	\centering
	\includegraphics[width=16cm]{Figure/Fig/Mandl-6/mutwise_hv.eps}
	\caption{Effect of mutation schemes on different algorithms for Mandl (6 routes). Each vertical bar represents the performance of a mutation scheme for an algorithm using HV value. For each algorithm we used five mutation schemes. So five vertical bars are grouped under each algorithm.}\label{fig:mandl6_mutwise_hv} 
\end{figure}

Next we move to larger datasets. Figure~\ref{fig:m0_mutwise_hv} summarizes mutation scheme wise HV results for Mumford0. We observe that for SPEA2, Guided Scheme achieves the best performance. Random Scheme III defeats others in case of MOEA/D. And for NSGAIII and $\theta$-DEA, Basic Scheme wins. Unlike previous cases, $\theta$-DEA clearly defeats others.  

\begin{figure} [!htbp]
	\centering
	\includegraphics[width=16cm]{Figure/Fig/M0-12/mutwise_hv.eps}
	\caption{Effect of mutation schemes on different algorithms for Mumford0. Each vertical bar represents the performance of a mutation scheme for an algorithm using HV value. For each algorithm we used five mutation schemes. So five vertical bars are grouped under each algorithm.}\label{fig:m0_mutwise_hv} 
\end{figure}

For Mumford1, Figure~\ref{fig:mandl4_mutwise_hv} shows the comparison of mutation schemes. Here we see Guided Scheme performs best for $\theta$-DEA while Random Scheme II performs best for the rest.  Similar to Mumford0, here $\theta$-DEA clearly defeats others.  

\begin{figure} [!htbp]
	\centering
	\includegraphics[width=16cm]{Figure/Fig/M1-15/mutwise_hv.eps}
	\caption{Effect of mutation schemes on different algorithms for Mumford1. Each vertical bar represents the performance of a mutation scheme for an algorithm using HV value. For each algorithm we used five mutation schemes. So five vertical bars are grouped under each algorithm.}\label{fig:m1_mutwise_hv} 
\end{figure}

Analyzing all these bar charts, following observations can be drawn:
\begin{itemize}
	\item The performance of a mutation scheme depends on the type of MaOEA.
	
	\item In most of the cases, Basic Scheme is beaten by other schemes. 

	\item For most of the MaOEAs, the best performing three schemes are among \{Basic Scheme, Random Scheme II, Guided Scheme\}.

	\item NSGAII and $\theta$-DEA jointly achieve the first place for smaller instances (Mandl-4, Mandl-6). But for larger instances $\theta$-DEA clearly beats other MaOEAs.
\end{itemize}

\subsection{Comparison with Previous Approach}

In this section we compare our MaOEA based approach with previous approaches. To accomplish this, we collected best solutions generated by several researchers. We also created a front for each MaOEA by combining all non-dominated solutions generated by that MaOEA. We list the number of non-dominated solutions generated by each MaOEA for each instance in Table~\ref{tab:rf_size}. Based on these resources, we evaluated two new metrics GDS and GAS, proposed in Section~\ref{sec:new_metric}, for each MaOEA.

% Table generated by Excel2LaTeX from sheet 'RF size'
\begin{table}[htbp]
	\centering
	\caption{Number of non-dominated solutions generated by each MaOEA}
	\begin{tabular}{|l|r|r|r|r|}
		\hline
		\multicolumn{1}{|c|}{\multirow{2}[4]{*}{Instance}} & \multicolumn{4}{c|}{No. of generated non-dominated solutions} \\
		\cline{2-5}    \multicolumn{1}{|c|}{} & \multicolumn{1}{c|}{SPEA2} & \multicolumn{1}{c|}{MOEA/D} & \multicolumn{1}{c|}{NSGAIII} & \multicolumn{1}{c|}{$\theta$-DEA} \\
		\hline
		Mandl-4 & 8212  & 787   & 9671  & 8290 \\
		\hline
		Mandl-6 & 10062 & 1612  & 12706 & 12909 \\
		\hline
		Mumford0 & 19488 & 4759  & 20437 & 19620 \\
		\hline
		Mumford1 & 16823 & 9717  & 18366 & 20918 \\
		\hline
		\hline
		Average & 13646 & 4219  & 15295 & 15434 \\
		\hline
	\end{tabular}%
	\label{tab:rf_size}%
\end{table}%

Table~\ref{tab:com_with_prev_mandl4} shows the comparative results for Mandl-4. In the leftmost column, we list the best solutions of Mandl-4 reported by 21 previous studies. Here for each MaOEA, we show GDS and GAS values with respect to these best solutions. We find that, solutions generated by SPEA2 dominated these best solutions in seven cases. NSGAIII solutions dominated 11 best solutions and $\theta$-DEA solutions dominated 13 best solutions. MOEA/D solutions failed to dominate any solution. In most of the cases, $\theta$-DEA generated more dominating solutions than SPEA2 and NSGAIII. And each MaOEA generated a large number of alternatives for all the best solutions which is evident from GAS values.

% Table generated by Excel2LaTeX from sheet 'mandl-4'
\begin{table}[!htbp]
	\centering
	\caption{Comparison with previous approaches for Mandl (4 routes)}
	\begin{tabular}{|l|r|r||r|r||r|r||r|r|}
		\hline
		\multicolumn{1}{|c|}{\multirow{2}[4]{*}{Reported best solution}} & \multicolumn{2}{c||}{SPEA2} & \multicolumn{2}{c||}{MOEA/D} & \multicolumn{2}{c||}{NSGAIII} & \multicolumn{2}{c|}{$\theta$-DEA} \\
		\cline{2-9}    \multicolumn{1}{|c|}{} & \multicolumn{1}{c|}{GDS} & \multicolumn{1}{c||}{GAS} & \multicolumn{1}{c|}{GDS} & \multicolumn{1}{c||}{GAS} & \multicolumn{1}{c|}{GDS} & \multicolumn{1}{c||}{GAS} & \multicolumn{1}{c|}{GDS} & \multicolumn{1}{c|}{GAS} \\
		\hline
		Mandl 1980~\cite{mandl1980evaluation} & 0     & 3388  & 0     & 601   & 4     & 6050  & 4     & 5584 \\
		\hline
		Kidwai 1998~\cite{kidwai1998optimal} & 1     & 4013  & 0     & 640   & 15    & 6738  & 10    & 6065 \\
		\hline
		Gundaliya 2000~\cite{gundaliya2000model} & 1     & 5703  & 0     & 753   & 6     & 8131  & 16    & 6979 \\
		\hline
		Chakroborty 2002~\cite{chakroborty2002optimal} & 4     & 6812  & 0     & 763   & 34    & 8998  & 141   & 7748 \\
		\hline
		Fan 2009 (Passenger)~\cite{fan2009simple} & 0     & 6919  & 0     & 763   & 0     & 9037  & 2     & 7817 \\
		\hline
		Fan 2009 (Operator)~\cite{fan2009simple} & 0     & 725   & 0     & 248   & 0     & 1686  & 0     & 2277 \\
		\hline
		Fan 2010~\cite{fan2010metaheuristic} & 2     & 7859  & 0     & 783   & 6     & 9567  & 63    & 8249 \\
		\hline
		Zhang 2010 (Passenger)~\cite{zhang2010multi} & 0     & 7268  & 0     & 775   & 10    & 9248  & 39    & 8040 \\
		\hline
		Zhang 2010 (Operator)~\cite{zhang2010multi} & 0     & 725   & 0     & 248   & 0     & 1686  & 0     & 2277 \\
		\hline
		Zhang 2010 (Compromise)~\cite{zhang2010multi} & 0     & 5308  & 0     & 730   & 3     & 7862  & 19    & 6767 \\
		\hline
		Mumford 2013 (Passenger)~\cite{mumford2013new} & 33    & 8128  & 0     & 783   & 90    & 9662  & 126   & 8272 \\
		\hline
		Mumford 2013 (Operator)~\cite{mumford2013new} & 0     & 725   & 0     & 248   & 0     & 1686  & 0     & 2277 \\
		\hline
		Nikoli{\'c} 2013~\cite{nikolic2013transit} & 4     & 7522  & 0     & 765   & 2     & 9400  & 33    & 8172 \\
		\hline
		Chew 2013 (Passenger)~\cite{chew2013genetic} & 0     & 6919  & 0     & 763   & 0     & 9037  & 2     & 7817 \\
		\hline
		Chew 2013 (Operator)~\cite{chew2013genetic} & 0     & 725   & 0     & 248   & 0     & 1686  & 0     & 2277 \\
		\hline
		Kechagiopoulos 2014~\cite{kechagiopoulos2014solving} & 46    & 8085  & 0     & 784   & 108   & 9648  & 388   & 8262 \\
		\hline
		Nikoli{\'c} 2014 (Passenger)~\cite{nikolic2014simultaneous} & 0     & 7703  & 0     & 782   & 0     & 9489  & 0     & 8206 \\
		\hline
		Nikoli{\'c} 2014 (Operator)~\cite{nikolic2014simultaneous} & 0     & 6330  & 0     & 754   & 0     & 8574  & 0     & 7411 \\
		\hline
		K{\i}l{\i}{\c{c}} 2014~\cite{kilicc2014demand} & 0     & 7333  & 0     & 776   & 1     & 9265  & 1     & 8071 \\
		\hline
		Nayeem 2014~\cite{nayeem2014transit} & 0     & 8212  & 0     & 787   & 0     & 9671  & 0     & 8290 \\
		\hline
		Rahman 2015~\cite{rahman2015transit} & 0     & 8212  & 0     & 787   & 0     & 9671  & 0     & 8286 \\
		\hline
		\hline
		\multicolumn{1}{|c|}{Average} & 4     & 5648  & 0     & 656   & 13    & 7466  & 40    & 6626 \\
		\hline
	\end{tabular}%
	\label{tab:com_with_prev_mandl4}%
\end{table}%

Next we compare our MaOEAs with previous studies in case of Mandl-6. Table~\ref{tab:com_with_prev_mandl6} summarizes this comparison where we list GDS and GAS values with respect to 23 reported best solutions in the same way. We find that SPEA2 can dominate nine solutions. NSGAIII dominated eight solutions. And $\theta$-DEA dominated 10 solutions. MOEA/D failed again to dominate any solution. GAS values show that, MaOEAs generated a large number of alternative solutions for each case. $\theta$-DEA continues to generate the largest number of dominating solutions in most of the cases.

% Table generated by Excel2LaTeX from sheet 'mandl-6'
\begin{table}[!htbp]
	\centering
	\caption{Comparison with previous approaches for Mandl (6 routes)}
	\begin{tabular}{|l|r|r||r|r||r|r||r|r|}
		\hline
		\multicolumn{1}{|c|}{\multirow{2}[4]{*}{Reported best solution}} & \multicolumn{2}{c||}{SPEA2} & \multicolumn{2}{c||}{MOEA/D} & \multicolumn{2}{c||}{NSGAIII} & \multicolumn{2}{c|}{$\theta$-DEA} \\
		\cline{2-9}    \multicolumn{1}{|c|}{} & \multicolumn{1}{c|}{GDS} & \multicolumn{1}{c||}{GAS} & \multicolumn{1}{c|}{GDS} & \multicolumn{1}{c||}{GAS} & \multicolumn{1}{c|}{GDS} & \multicolumn{1}{c||}{GAS} & \multicolumn{1}{c|}{GDS} & \multicolumn{1}{c|}{GAS} \\
		\hline
		Baaj 1991~\cite{baaj1991ai} & 8     & 8590  & 0     & 1603  & 20    & 11773 & 23    & 11971 \\
		\hline
		Shih 1994~\cite{shih1994design} & 0     & 7037  & 0     & 1557  & 6     & 10976 & 4     & 11477 \\
		\hline
		Kidwai 1998~\cite{kidwai1998optimal} & 4     & 7226  & 0     & 1552  & 9     & 11076 & 24    & 11536 \\
		\hline
		Gundaliya 2000~\cite{gundaliya2000model} & 0     & 8573  & 0     & 1599  & 0     & 11832 & 0     & 11937 \\
		\hline
		Fan 2009 (Passenger)~\cite{fan2009simple} & 0     & 8237  & 0     & 1576  & 0     & 11731 & 0     & 11808 \\
		\hline
		Fan 2009 (Operator)~\cite{fan2009simple} & 0     & 395   & 0     & 206   & 0     & 874   & 0     & 1970 \\
		\hline
		Fan 2010~\cite{fan2010metaheuristic} & 4     & 9859  & 0     & 1612  & 6     & 12568 & 20    & 12816 \\
		\hline
		Zhang 2010 (Passenger)~\cite{zhang2010multi} & 199   & 10002 & 0     & 1612  & 282   & 12677 & 260   & 12897 \\
		\hline
		Zhang 2010 (Operator)~\cite{zhang2010multi} & 0     & 395   & 0     & 206   & 0     & 874   & 0     & 1970 \\
		\hline
		Zhang 2010 (Compromise)~\cite{zhang2010multi} & 0     & 3241  & 0     & 1336  & 0     & 7095  & 0     & 7628 \\
		\hline
		Mumford 2013 (Passenger)~\cite{mumford2013new} & 3     & 9964  & 0     & 1612  & 0     & 12666 & 13    & 12885 \\
		\hline
		Mumford 2013 (Operator)~\cite{mumford2013new} & 0     & 260   & 0     & 163   & 0     & 482   & 0     & 1405 \\
		\hline
		Nikoli{\'c} 2013~\cite{nikolic2013transit} & 1     & 9975  & 0     & 1612  & 1     & 12676 & 0     & 12889 \\
		\hline
		Chew 2013 (Passenger)~\cite{chew2013genetic} & 0     & 8237  & 0     & 1576  & 0     & 11731 & 0     & 11808 \\
		\hline
		Chew 2013 (Operator)~\cite{chew2013genetic} & 0     & 395   & 0     & 206   & 0     & 874   & 0     & 1970 \\
		\hline
		Majima 2014~\cite{majima2014application} & 0     & 7206  & 0     & 1565  & 0     & 11137 & 0     & 11507 \\
		\hline
		Owais 2014~\cite{owais2014simple} & 16    & 9859  & 0     & 1612  & 24    & 12572 & 54    & 12819 \\
		\hline
		Kechagiopoulos 2014~\cite{kechagiopoulos2014solving} & 1     & 9778  & 0     & 1612  & 2     & 12505 & 2     & 12749 \\
		\hline
		Nikoli{\'c} 2014 (Passenger)~\cite{nikolic2014simultaneous} & 0     & 9216  & 0     & 1603  & 0     & 12162 & 3     & 12294 \\
		\hline
		Nikoli{\'c} 2014 (Operator)~\cite{nikolic2014simultaneous} & 0     & 7679  & 0     & 1572  & 0     & 11483 & 0     & 11659 \\
		\hline
		K{\i}l{\i}{\c{c}} 2014~\cite{kilicc2014demand} & 1     & 9697  & 0     & 1609  & 0     & 12464 & 1     & 12694 \\
		\hline
		Nayeem 2014~\cite{nayeem2014transit} & 0     & 10062 & 0     & 1612  & 0     & 12704 & 0     & 12909 \\
		\hline
		Rahman 2015~\cite{rahman2015transit} & 0     & 10059 & 0     & 1612  & 0     & 12704 & 0     & 12909 \\
		\hline
		\hline
		\multicolumn{1}{|c|}{Average} & 10    & 7215  & 0     & 1340  & 15    & 9897  & 18    & 10283 \\
		\hline
	\end{tabular}%
	\label{tab:com_with_prev_mandl6}%
\end{table}%

 Now we perform the comparison for Mumford0. As it was published few years back (2013), we were able to collect only four reported best solutions. The comparative results are shown in Table~\ref{tab:com_with_prev_m0}. Here SPEA2 can dominate one solution while both NSGAIII and $\theta$-DEA can dominate three solutions. In all the cases, $\theta$-DEA generates the highest number of dominating solutions. From GAS values we see that, each MaOEA can generate a large number of alternatives to each best solution.

% Table generated by Excel2LaTeX from sheet 'm0'
\begin{table}[!htbp]
	\centering
	\caption{Comparison with previous approaches for Mumford0}
	\begin{tabular}{|l|r|r||r|r||r|r||r|r|}
		\hline
		\multicolumn{1}{|c|}{\multirow{2}[4]{*}{Reported best solution}} & \multicolumn{2}{c||}{SPEA2} & \multicolumn{2}{c||}{MOEA/D} & \multicolumn{2}{c||}{NSGAIII} & \multicolumn{2}{c|}{$\theta$-DEA} \\
		\cline{2-9}    \multicolumn{1}{|c|}{} & \multicolumn{1}{c|}{GDS} & \multicolumn{1}{c||}{GAS} & \multicolumn{1}{c|}{GDS} & \multicolumn{1}{c||}{GAS} & \multicolumn{1}{c|}{GDS} & \multicolumn{1}{c||}{GAS} & \multicolumn{1}{c|}{GDS} & \multicolumn{1}{c|}{GAS} \\
		\hline
		Mumford 2013 (Passenger)~\cite{mumford2013new} & 9     & 19488 & 0     & 4759  & 205   & 20437 & 306   & 19620 \\
		\hline
		Mumford 2013 (Operator)~\cite{mumford2013new} & 0     & 1797  & 0     & 1204  & 0     & 1639  & 0     & 1658 \\
		\hline
		K{\i}l{\i}{\c{c}} 2014~\cite{kilicc2014demand} & 0     & 19448 & 0     & 4759  & 31    & 20422 & 118   & 19611 \\
		\hline
		Nayeem 2014~\cite{nayeem2014transit} & 0     & 19488 & 0     & 4759  & 6     & 20437 & 14    & 19620 \\
		\hline
		\hline
		\multicolumn{1}{|c|}{Average} & 2     & 15055 & 0     & 3870  & 61    & 15734 & 110   & 15127 \\
		\hline
	\end{tabular}%
	\label{tab:com_with_prev_m0}%
\end{table}%

Finally we show the comparative results for Mumford1 in Table~\ref{tab:com_with_prev_m1}. For the same reason as with Mumford0, we were able to collect only five reported best solutions. SPEA2 dominated in one case. NSGAIII and $\theta$-DEA dominated in two cases. However, for each best solution, each MaOEA generated a large number of alternatives.

% Table generated by Excel2LaTeX from sheet 'm1'
\begin{table}[!htbp]
	\centering
	\caption{Comparison with previous approaches for Mumford1}
	\begin{tabular}{|l|r|r||r|r||r|r||r|r|}
		\hline
		\multicolumn{1}{|c|}{\multirow{2}[4]{*}{Reported best solution}} & \multicolumn{2}{c||}{SPEA2} & \multicolumn{2}{c||}{MOEA/D} & \multicolumn{2}{c||}{NSGAIII} & \multicolumn{2}{c|}{$\theta$-DEA} \\
		\cline{2-9}    \multicolumn{1}{|c|}{} & \multicolumn{1}{c|}{GDS} & \multicolumn{1}{c||}{GAS} & \multicolumn{1}{c|}{GDS} & \multicolumn{1}{c||}{GAS} & \multicolumn{1}{c|}{GDS} & \multicolumn{1}{c||}{GAS} & \multicolumn{1}{c|}{GDS} & \multicolumn{1}{c|}{GAS} \\
		\hline
		Mumford 2013 (Passenger)~\cite{mumford2013new} & 12    & 16719 & 0     & 9717  & 81    & 18366 & 155   & 20916 \\
		\hline
		Mumford 2013 (Operator)~\cite{mumford2013new} & 0     & 1737  & 0     & 5942  & 0     & 4572  & 0     & 5175 \\
		\hline
		K{\i}l{\i}{\c{c}} 2014~\cite{kilicc2014demand} & 0     & 16580 & 0     & 9717  & 35    & 18354 & 70    & 20913 \\
		\hline
		Nayeem 2014~\cite{nayeem2014transit} & 0     & 12025 & 0     & 9715  & 0     & 15538 & 0     & 19006 \\
		\hline
		Rahman 2015~\cite{rahman2015transit} & 0     & 12003 & 0     & 9715  & 0     & 15378 & 0     & 18916 \\
		\hline
		\hline
		\multicolumn{1}{|c|}{Average} & 2     & 11813 & 0     & 8961  & 23    & 14442 & 45    & 16985 \\
		\hline
	\end{tabular}%
	\label{tab:com_with_prev_m1}%
\end{table}%

Now we compare MaOEAs based on average GDS and GAS found from Tables~\ref{tab:com_with_prev_mandl4} - \ref{tab:com_with_prev_m1}. Figure~\ref{fig:algowise_gds_gas} visualizes these average values. We find that according to GDS, $\theta$-DEA clearly beats other MaOEAs in all instances. While considering GAS, $\theta$-DEA beats others for Mandl-6 and Mumford1. For other two instances, NSGAIII achieves the highest value.

\begin{figure}[!htbp]
	\centering
	\begin{subfigure}[b]{0.51\textwidth}
		\includegraphics[width=\textwidth]{Figure/algowise_gds}
		%\caption{pr03}
		%\label{fig:con_pr03}
	\end{subfigure}%
	\begin{subfigure}[b]{0.51\textwidth}
		\includegraphics[width=\textwidth]{Figure/algowise_gas}
		%\caption{pr06}
		%\label{fig:con_pr06}
	\end{subfigure}%
	\caption{Comparison of MaOEAs based on average GDS and GAS.}
	\label{fig:algowise_gds_gas}
\end{figure}

A major challenge for any algorithm that tackles TNDP is to maintain its performance with the increase of dataset size.  We find several proposed algorithms behaving differently for larger datasets. So it is necessary to observe the behavior of our MaOEAs while increasing network size.  To quantify this behavior we define Adjusted GAS for each problem instance $p$ and MaOEA $m$ as follows:
\begin{equation}
\text{Adjusted GAS}(p, m)  = \frac{ \text{Average GAS}(p, m) }{ \text{No. of non-dominated solution generated by } m \text{ for } p }
	\label{eq:adjusted_gas}
\end{equation}
Adjusted GAS estimate the percentage of competitive solutions generated by a MaOEA for a problem instance with respect to best solutions generated by previous approaches. A good MaOEA should achieve similar Adjusted GAS across different instances. Figure~\ref{fig:adjusted_gas} presents problem instance wise Adjusted GAS values. Here the instances are ordered according to their sizes. It is evident from this figure that all the MaOEAs behave in similar way throughout all the instances. Even some of their behavior improves for larger instances. Alike previous scenarios, here we find NSGAIII and $\theta$-DEA as top performers. They behave almost uniformly for all the instances achieving Adjusted GAS around 0.8.
\begin{figure} [!htbp]
	\centering
	\includegraphics[width=12cm]{Figure/adjusted_gas}\\
	\caption{Response of MaOEAs to the increase in instance size.}\label{fig:adjusted_gas} 
\end{figure}
%\section{Discussion}
%
%For Plan 3, we used only the LLGC algorithm. The results are shown in graph form in Figure~\ref{RipandLLGC}[d-f]. Like Plan 2, the performance of the multiclass version of LLGC performed the worst. BBO framework also performed better than the binarized version of LLGC. So, we can come to the conclusion that binarization can be an effective approach to semi-supervised classification and BBO framework can help improve the results even more. Again the standard deviation was not very high. Unlike Plan 2 though, the performance of all the classifiers remained almost the same for different fractions of labeled data. This is a property of the LLGC framework. Its performance varies very little with labeled data as can be seen from~\cite{DBLP:conf/nips/ZhouBLWS03}. Boosting also performs similarly to the BBO framework. Furthermore, like Plan 2 the BBO framework performs worse than the binarized version of LLGC for the some fixed data sets irrespective of the percentage of labeled data. This again suggests that the underlying properties of the data set may impact the performance of the BBO framework.   

\section{Summary}
%This chapter presented the evaluation of our MaOEAs’ performance on several road networks of different sizes. With the evaluation measures and scenarios considered in this study, we observed the advantages of the MaOEAs over previous approaches. Among the MaOEAs, $\theta$-DEA proved to be the most robust and consistent method. 

In this chapter we evaluated our proposed methodology by simulating on well-known datasets and analyzing obtained results. While examining approximated PFs for the problem instances, we found trade-off relationship among objective functions. This justifies our choice of evolutionary many-objective approach to tackle TNDP. Next we experimented with crossover and observed that it can improve performance if the rate is properly tuned based on the nature of dataset and mutation scheme. 


We also compared the performance of different mutation schemes. We found that for most of the cases Basic Scheme is outperformed by other schemes and the top three schemes are among \{Basic Scheme, Random Scheme II, Guided Scheme\}. At the end we compared our MaOEAs with previous approaches based on  GDS and GAS. Although it is very difficult to dominate a solution in high dimensional space, our MaOEAs dominated a good number of previous best solutions. GAS values clearly demonstrated the superiority of our approach. Considering all the observations, $\theta$-DEA proved to be the most robust and consistent MaOEA. 

%We now summarize our findings from the previous section. At first we examined the approximated PFs of four TNDP instances. We found trade-off relationship between different pairs of objective functions. Also we noticed that the objective functions are differently scaled. There were no apparent redundancy among the information captured by different objectives. So, evolutionary many-objective approach is a reasonable choice to tackle TNDP.
%
%Next we experimented with different crossover rates. We observed that for most of the cases crossover can improve algorithm performance. But the crossover rate should be tuned carefully based on the nature of problem instance and mutation scheme. The MaOEA which showed the most consistent behavior across all the instances and mutation schemes is $\theta$-DEA. 
%
%We also compared the performance of different mutation schemes and found that the best performing scheme varied across the algorithms and problem instances. But for most of the cases, the best three schemes are among \{Basic Scheme, Random Scheme II, Guided Scheme\}. At the same time we noticed that for most of the cases Basic Scheme is outperformed by other schemes. Considering algorithm, both NSGAIII and $\theta$-DEA performed best for smaller instances. However, $\theta$-DEA clearly beat others for larger instances.
%
%Finally we compared our MaOEAs with previous approaches based on two proposed metrics GDS and GAS. Although it is very difficult to dominate a solution in high dimensional objective space, we found that SPEA2, NSGAIII and $\theta$-DEA can dominate a good number of previous best solutions. When we performed the comparison based on GAS, the superiority of our approach became obvious. For each reported best solution all the MaOEAs generated a large number of alternative solutions. We also captured the MaOEAs' response in terms of Adjusted GAS while varying instance size. We established that the response did not vary considerably. Among the MaOEAs, we found $\theta$-DEA to be the most robust and consistent method. 



