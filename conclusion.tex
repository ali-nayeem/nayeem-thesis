\chapter{Conclusion}

\label{conclusion}

%\section{Conclusion}

In this modern age, an effective tool for solving TNDP is crucial for improving urban environment \& economy. TNDP involves different stakeholders with conflicting interests. Moreover their interests can be of multiple dimensions to meet modern challenges. As such TNDP is essentially a many-objective optimization problem. Yet most of the researchers neglected this issue and thus failed to provide a diverse set of alternative solutions to transport planners.

In this study, we constructed a new formulation for the many-objective TNDP to meet real-world requirements. We developed problem specific genetic operators which can be used with any evolutionary techniques to tackle TNDP. MOEA is a popular approach to solve realistic optimization problems. However, it struggles to achieve its goal when dealing with high dimensional objective space. That's why MaOEA comes into the scene. We adapted several state-of-the-art MaOEAs to solve TNDP using our genetic operators. 

We performed extensive experiments to evaluate the performance of our proposed methodology. We used Mandl's benchmark network along with two larger networks Mumford0 and Mumford1 as datasets. We constructed approximated PF to find the correlation among different objective functions. We analyzed the effect of different components of our MaOEA based framework using HV metric. To compare with previous approaches we collected best solutions generated by several well-known research works. We calculated objective vectors for those solutions using our evaluation model and examined the advantages of MaOEAs over those vectors based on two proposed metrics: GDS and GAS. The most important thing that can be noted from the acquired experimental results is that our framework generates a large number of alternative solutions for each of the previously reported solution. It can also be seen that many of the previous solutions are dominated by our solutions. We also observed that our framework performs similarly independent of the dataset size. From these observations we can conclude that, our MaOEA based framework is more effective than the existing methods to address modern challenges.


\section{Future Direction}

In this study, we focus on developing an effective optimization approach to solve TNDP. However we do not consider its efficiency. The \textit{effectiveness} of an optimization algorithm refers to the quality of the solutions found or its robustness in finding desired solutions. The \textit{efficiency} characterizes the runtime behavior of the optimization algorithm. Along with efficiency, some other issues need to be considered for enhancing our current contribution. They are described as follows:

\begin{itemize}
	\item 	As the size and complexity of problem instances increase, the running time of any population based algorithm for solving TNDP like ours become unacceptable. The computational bottleneck is the time required to evaluate the objective functions of a solution. In future we can implement parallel models of MaOEAs to distribute the computational workload among multiple execution units.
	
	\item In this study we generate a fixed set of well-distributed reference points to guide the search directions of MaOEAs. If we can devise a generic way to integrate transport planners' preference in terms of a small set of objective vectors, we can reduce search directions as well as running time substantially. 
	 
	\item Currently our framework generates a set of alternative solutions for transport planners. To assist transport planners in choosing the ultimate solution, we can integrate Multi Criteria Decision Making (MCDM) methods into our framework for reducing the number of alternative solutions to a manageable size.
	
	\item At present we follow the common assumption that travel time and waiting time are independent of congestion effect. We can investigate the ways to integrate this effect into our evaluation model.
	
	\item We can refine our theoretical model by consulting transport planners and determine the applicability of our MaOEAs to real-world problems. We can also investigate how our problem-solving techniques can be integrated into commercial software toolkits such as VISUM, Emme/3.
\end{itemize}

\begin{figure}[h]
	\centering
	\begin{subfigure}[b]{0.4 \textwidth}
		\includegraphics[width=\linewidth]{Figure/visum.jpg}
		\caption{VISUM} \label{fig:visum}
	\end{subfigure}
	\hspace*{0.6cm} % separation between the subfigures
	\begin{subfigure}[b]{0.5\textwidth}
		\includegraphics[width=\linewidth]{Figure/emme.jpg}
		\caption{Emme/3} \label{fig:emme}
	\end{subfigure}
	\caption{Commercial transport planning softwares.} \label{fig:transport_software}
\end{figure}
