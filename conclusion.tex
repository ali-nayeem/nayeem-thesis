\chapter{Conclusion}\label{ch:conclusion}

In this chapter, we conclude this thesis by summarizing our major contributions while highlighting the notable findings that can be considered as the main takeaway of this thesis. Then we outline possible directions for future extension.

\section{Summary}
We introduced a phylogeny-aware multi-objective optimization approach to compute MSA with an ultimate goal to infer the phylogenetic tree from the resultant alignments. To optimize MSA, we proposed two simple objective functions in addition to the existing ones. We judged the potential capability of each objective function to yield better trees by employing domain knowledge as well as by applying statistical approaches. We employed multiple linear regression to measure the degree of association between the individual objective functions and the quality of inferred phylogenetic tree (i.e., FN rate). Thus, we provide empirical justification to choose two bi-objective formulations to move forward. Afterwards, we performed extensive experimentation with both simulated and biological datasets to demonstrate the benefit of our approach. We showed that the simultaneous optimization of a set of phylogeny-aware objective functions can lead to phylogenetic trees with improved accuracy than that of the state-of-the-art MSA tools. From this finding, we would like to hypothesize that, the use of domain specific measures can aid an MSA methods in other application domain as well.

Afterwards we focused on inferring better phylogenetic trees from MSAs by incorporating many application-aware objective functions through decomposition-based MO principles. In this direction, we developed the PMAO framework, which is based on PASTA, one of the most celebrated algorithms/tools in this regard. We evaluated the PMAO framework and examined its capability to yield high-quality tress by experimenting on the widely-used BAliBASE 3.0 benchmark. The PMAO framework, like other multi-objective algorithms, outputs a good number of alternative solutions that are equivalent (usually referred to as the non-dominated solutions) in the context of the conflicting objectives considered in the MO framework. Some of these solutions may however be of relatively lower quality from the application perspective. To this end, we innovatively employed machine learning to help PMAO generate only five solutions encompassing at least one top quality solution. Furthermore, summarizing those five solutions to obtain a single one offered better accuracy over PASTA in most of the cases, although not as good as the overall PMAO best.

Next, we presented MAMMLE, a framework through which we infuse the concept of MO application-awareness into MUSCLE by incorporating our identified four application-aware objectives within the iterative refinement phase thereof through an MO strategy. MAMMLE generates multiple alternative alignments and for each of them an ML tree is inferred. We took these multiple hypotheses into our advantage and developed an ensemble approach for producing a better phylogenetic tree. MAMMLE offered a significant improvement (upto 27\% in our experiments) in tree accuracy over MUSCLE. We implemented our overall approach for phylogeny estimation from unaligned sequences as a flexible framework whose components can potentially be modified, replaced or further refined by bioinformatics researchers and practitioners.

Finally, we presented an evolutionary multi-objective optimization algorithm, namely, SNOGA. SNOGA is a modified version of the popular NSGA-II, which combined various optimization criteria to find a suitable search space containing highly accurate species trees. Our experimental results on a collection of simulated datasets demonstrated that the multi-objective approach can lead us to a tree-space containing significantly better trees than the trees estimated by ASTRAL and MP-EST which are two of the most widely used methods. 

\section{Future Directions}

\begin{itemize}
	\item we will employ recently developed MO algorithms (e.g.,~\cite{8981871, 9047876, 9097242}) to find out which one can give us the best performance.
	
	\item But it is possible to achieve enough speedup by utilizing advanced computing architectures as well as efficient implementation of core operations~\cite{8255834, 7738460}.  
	 
	\item we endeavor to enhance the summarizing approach and apply other strategies of leveraging the alternative MSAs/trees to single out the overall PMAO best solution.s
	
	\item However, the cases (45 out of 147) where MUSCLE is better (within 15\%) demands further research effort to enhance the current ensemble method (i.e., greedy consensus).
	
	\item At present, our mutation selects one from NNI/SPR/TBR at random with equal probability. As a future improvement, we will improve it by adaptively adjusting the selection probabilities based on the success rate of an operator in the previous generation. Also, we are planning to embed domain knowledge inside NNI, SPR and TBR so that they can make informed (as opposed to random) rearrangement in the given tree. 
	
	\item To enable SNOGA processing large datasets within a reasonable time, we will improve the efficiency of objective evaluations. Moreover, we are adapting a popular decomposition based EMO algorithm, namely, MOEA/D~\cite{zhang2007moea}, to effectively solve this problem.
	
	\item Throughout the
	thesis, we used only maximum likelihood for estimating gene trees, and only
	a handful of summary methods for estimating the species tree. Other variations of the pipeline might lead to different patterns of performance. For
	example, gene trees could be estimated using Bayesian methods instead of
	maximum likelihood, 
\end{itemize}


%In this modern age, an effective tool for solving TNDP is crucial for improving urban environment \& economy. TNDP involves different stakeholders with conflicting interests. Moreover their interests can be of multiple dimensions to meet modern challenges. As such TNDP is essentially a many-objective optimization problem. Yet most of the researchers neglected this issue and thus failed to provide a diverse set of alternative solutions to transport planners.
%
%In this study, we constructed a new formulation for the many-objective TNDP to meet real-world requirements. We developed problem specific genetic operators which can be used with any evolutionary techniques to tackle TNDP. MOEA is a popular approach to solve realistic optimization problems. However, it struggles to achieve its goal when dealing with high dimensional objective space. That's why MaOEA comes into the scene. We adapted several state-of-the-art MaOEAs to solve TNDP using our genetic operators. 
%
%We performed extensive experiments to evaluate the performance of our proposed methodology. We used Mandl's benchmark network along with two larger networks Mumford0 and Mumford1 as datasets. We constructed approximated PF to find the correlation among different objective functions. We analyzed the effect of different components of our MaOEA based framework using HV metric. To compare with previous approaches we collected best solutions generated by several well-known research works. We calculated objective vectors for those solutions using our evaluation model and examined the advantages of MaOEAs over those vectors based on two proposed metrics: GDS and GAS. The most important thing that can be noted from the acquired experimental results is that our framework generates a large number of alternative solutions for each of the previously reported solution. It can also be seen that many of the previous solutions are dominated by our solutions. We also observed that our framework performs similarly independent of the dataset size. From these observations we can conclude that, our MaOEA based framework is more effective than the existing methods to address modern challenges.
%
%
%\section{Future Direction}
%
%In this study, we focus on developing an effective optimization approach to solve TNDP. However we do not consider its efficiency. The \textit{effectiveness} of an optimization algorithm refers to the quality of the solutions found or its robustness in finding desired solutions. The \textit{efficiency} characterizes the runtime behavior of the optimization algorithm. Along with efficiency, some other issues need to be considered for enhancing our current contribution. They are described as follows:
%
%\begin{itemize}
%	\item 	As the size and complexity of problem instances increase, the running time of any population based algorithm for solving TNDP like ours become unacceptable. The computational bottleneck is the time required to evaluate the objective functions of a solution. In future we can implement parallel models of MaOEAs to distribute the computational workload among multiple execution units.
%	
%	\item In this study we generate a fixed set of well-distributed reference points to guide the search directions of MaOEAs. If we can devise a generic way to integrate transport planners' preference in terms of a small set of objective vectors, we can reduce search directions as well as running time substantially. 
%	 
%	\item Currently our framework generates a set of alternative solutions for transport planners. To assist transport planners in choosing the ultimate solution, we can integrate Multi Criteria Decision Making (MCDM) methods into our framework for reducing the number of alternative solutions to a manageable size.
%	
%	\item At present we follow the common assumption that travel time and waiting time are independent of congestion effect. We can investigate the ways to integrate this effect into our evaluation model.
%	
%	\item We can refine our theoretical model by consulting transport planners and determine the applicability of our MaOEAs to real-world problems. We can also investigate how our problem-solving techniques can be integrated into commercial software toolkits such as VISUM, Emme/3.
%\end{itemize}
%
%\begin{figure}[h]
%	\centering
%	\begin{subfigure}[b]{0.4 \textwidth}
%		\includegraphics[width=\linewidth]{Figure/visum.jpg}
%		\caption{VISUM} \label{fig:visum}
%	\end{subfigure}
%	\hspace*{0.6cm} % separation between the subfigures
%	\begin{subfigure}[b]{0.5\textwidth}
%		\includegraphics[width=\linewidth]{Figure/emme.jpg}
%		\caption{Emme/3} \label{fig:emme}
%	\end{subfigure}
%	\caption{Commercial transport planning softwares.} \label{fig:transport_software}
%\end{figure}
